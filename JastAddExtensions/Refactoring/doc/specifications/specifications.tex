\documentclass{article}

\usepackage{amsmath}
\usepackage{xspace}
\usepackage{xcolor}
\usepackage{stmaryrd}
\usepackage{algorithm}
\usepackage{algorithmic}
\usepackage{listings}
\usepackage{supertabular}
\usepackage{multiletter}
\usepackage{morefloats}

\title{Specifications of Implemented Refactorings}
\author{Max Sch\"afer, Tom\'a\v{s} Ko\v{c}isk\'y}

\newcommand{\refactoring}[1]{\textsc{#1}}
\newcommand{\refactoringNoExt}[1]{\lfloor\refactoring{#1}\rfloor}
\newcommand{\type}[1]{\ensuremath{\text{\textsl{#1}}}}
\newcommand{\util}[1]{\ensuremath{\text{\texttt{#1}}}}
\newcommand{\orelse}{\,\textbf{or}\,}
\newcommand{\bind}{\gg=}
\newcommand{\assert}{\textbf{assert}\,\,}
\newcommand{\locked}[1]{\ensuremath{\llbracket #1\rrbracket}}
\newcommand{\option}[1]{\ensuremath{\text{\texttt{option}\xspace #1}}}
\newcommand{\None}{\util{None}\xspace}
\newcommand{\Some}[1]{\util{Some}\xspace #1}
\newcommand{\listtp}[1]{\ensuremath{\text{\texttt{list}\xspace #1}}}
\newcommand{\settp}[1]{\ensuremath{\text{\texttt{set}\xspace #1}}}
\newcommand{\sourcelink}[1]{\texttt{#1}}

\definecolor{KWColor}{rgb}{0.5,0,0.67}

\lstset{
  language=[JastAdd]Java,
  basicstyle=\ttfamily\small,
  commentstyle=\footnotesize\rmfamily\emph,
  keywordstyle=\bf\ttfamily\small\color{KWColor},
  morekeywords={with},
  escapeinside={/*@}{@*/},
  literate={[}{{$\lfloor$}}1 {]}{{$\rfloor$}}1,
}

\newcommand{\code}[1]{\lstinline$#1$}
\newcommand{\progoutput}[1]{\texttt{#1}}
\lstnewenvironment{java}{}{}
\newcommand{\kw}[1]{\textbf{\color{KWColor}{#1}}}

\begin{document}
\maketitle

This document collects the pseudo-code specifications of all refactoring implemented in our engine. \textbf{Note:} This is work in progress; some specifications are missing, and not all implementations agree completely with the specifications.

\section{Pseudocode Conventions}
We give our specifications in generic, imperative pseudocode. Parameters and return values are informally typed, with syntax tree nodes having one of the types from Fig.~\ref{fig:node types}. Additionally, we use an ML-like \util{option} type with constructors \util{None} and \util{Some} for functions that may or may not return a value.

Where convenient, we make use of ML-like lists, with list literals of the form $[1; 2; 3]$ and $|xs|$ indicating the length of list $xs$.

The names of refactorings are written in \refactoring{small caps}, whereas utility functions appear in \util{monospace}. A list of utility functions with brief descriptions is given in Fig.~\ref{fig:utility}. An invocation of a refactoring is written with floor-brackets $\refactoringNoExt{like this}()$ to indicate that any language extensions used in the output program produced by the refactoring should be eliminated before proceeding.

We write $A<:B$ to mean that type $A$ extends or implements type $B$, and $m<:m'$ to mean that method $m$ overrides method $m'$.

\section{The Refactorings}

\subsection{\refactoring{Convert Anonymous to Local}}
This refactoring converts an anonymous class to a local class. Implemented in \sourcelink{TypePromotion/AnonymousClassToLocalClass.jrag}.

\begin{algorithm}
\caption{$\refactoring{Convert Anonymous to Local}(A : \type{AnonymousClass}, n : \type{Name}) : \type{LocalClass}$}
\begin{algorithmic}[1]
\REQUIRE Java
\ENSURE Java $\cup$ locked names
\medskip
\STATE $c \leftarrow \util{getClassInstanceExpr}(A)$
\STATE $s \leftarrow [\refactoring{Extract Temp}](c, \util{unCapitalise}(n))$
\STATE $b \leftarrow \util{enclosingBodyDecl}(s)$
\STATE $\util{lockTypeNames}(b, n)$
\STATE $t \leftarrow \util{asNamedClass}(A, n)$
\STATE $\util{removeTypeDecl}(c)$
\STATE $\util{setTypeAccess}(c, \locked{t})$
\RETURN $\util{insertLocalClass}(s, t)$
\end{algorithmic}
\end{algorithm}

We first retrieve the class instance expression $c$ of which $A$ is a part. Then we apply the \code{Extract Temp} refactoring to move $c$ into its own statement. All references to types named $n$ are locked within the enclosing body declaration $b$. Then $A$ is converted into a class $t$ with name $n$. We remove $A$ from $c$, make sure that $c$ constructs an object of type $t$, and insert $t$ as a local class right before the statement containing $c$.

\subsection{\refactoring{Convert Anonymous to Nested}}
This refactoring converts an anonymous class to a member class. Implemented in \sourcelink{TypePromotion/AnonymousClassToMemberClass.jrag}; see Algorithm~\ref{alg:ConvertAnonymousToNested}.

\begin{algorithm}
\caption{$\refactoring{Convert Anonymous to Nested}(A : \type{AnonymousClass}) : \type{MemberType}$}\label{alg:ConvertAnonymousToNested}.
\begin{algorithmic}[1]
\REQUIRE Java
\ENSURE Java
\medskip
\STATE $L \leftarrow \refactoring{Convert Anonymous to Local}(A)$
\RETURN $\refactoring{Convert Local to Member Class}(L)$
\end{algorithmic}
\end{algorithm}

Note: the implementation additionally handles the case where $A$ occurs in a field initialiser.

\subsection{\refactoring{Convert Local to Member Class}}
This refactoring converts a local class to a member class. Implemented in \sourcelink{TypePromotion/LocalClassToMemberClass.jrag}.

\begin{algorithm}
\caption{$\refactoring{Convert Local to Member Class}(L : \type{LocalClass}) : \type{MemberType}$}
\begin{algorithmic}[1]
\REQUIRE Java
\ENSURE Java $\cup$ locked names, fresh variables
\medskip
\STATE $h \leftarrow \util{enclosingType}(L)$
\STATE $\util{closeOverTypeVariables}(L)$
\STATE $\util{closeOverLocalVariables}(L)$
\IF{$\util{inStaticContext}(L)$}
  \STATE $\util{addModifier}(L, \text{\code{static}})$
\ENDIF
\STATE $\util{lockTypeNames}(\util{programRoot}(), \util{name}(L))$
\STATE $\util{lockNames}(L)$
\STATE $\util{removeStmt}(L)$
\RETURN $\util{insertMemberType}(h, L)$
\end{algorithmic}
\end{algorithm}

We start by computing the enclosing type $h$ of $L$, into which we want to eventually insert $L$ as a member type. Then we close $L$ over type variables and local variables from the enclosing body declaration.

The utility function $\util{closeOverTypeVariables}(L)$ collects all type variables $V$ of the enclosing body declaration of $L$ which are used inside $L$. Every such $L$ is added as a type parameter to $L$, and every use of $L$ is augmented by a corresponding type access.

Similarly, $\util{closeOverLocalVariables}(L)$ adds a field $f_v$ to $L$ for every local variable $v$ of the enclosing body declaration of $L$. All uses of $v$ from within $L$ are replaced by locked accesses to $f_v$, and the constructors and instantiations of $L$ are adjusted to initialise $f_v$ to the value of $v$.

If $L$ is in a static context (for example because its enclosing body declaration is static), a \code{static} qualifier is added. All references to types with the same name as $L$ throughout the program are locked; likewise, all names within $L$ are locked. Then $L$ is removed from its enclosing body declaration and inserted as a member type into $h$.

\subsection{\refactoring{Extract Class}}
This refactoring extracts some fields of a class into a newly created member class. Implemented in \sourcelink{ExtractClass/ExtractClass.jrag}; see Algorithm~\ref{alg:ExtractClass}.

Initializer evaluation is order independent for example
if we can invert their order without breaking name and dataflow dependencies.


\begin{algorithm}
\caption{$\refactoring{Extract Class}(C : \type{Class}, fs : \listtp{\type{Field}}, n : \type{Name}, fn : \type{Name})$}\label{alg:ExtractClass}
\begin{algorithmic}[1]
\REQUIRE Java
\ENSURE Java $\cup$ locked names, locked dataflow, first-class array init
\medskip
\STATE $v \leftarrow \text{maximum visibility of any of the $fs$}$
\STATE $W \leftarrow \text{new \code{static} class of name $n$ with visibility $v$}$
\STATE $\refactoring{Insert Type}(C, W)$
\STATE $w \leftarrow \text{new field of type $W$ and name $fn$, initialised to a new instance of $W$}$
\STATE $\refactoring{Insert Field}(C, w)$
\FORALL{$f\in fs$}
  \STATE \assert $f$ is not static
  \FORALL{uses $v$ of $f$}
    \STATE qualify $v$ with a locked access to $w$
  \ENDFOR
  \IF{$f$ has initialiser}
    \STATE split field declaration and initializer, leaving initializer in initializer block after
  \ENDIF
  \STATE remove $f$
  \STATE $\refactoring{Insert Field}(W, f)$
\ENDFOR
\STATE $inits \leftarrow \{\text{initializers of }fs\}$
\FORALL{$init\in inits$}
  \STATE lock names and dataflow
  \STATE move $init$ after already moved initializers (possibly $w$)
  \STATE \code{try} unlocking names and dataflow
  \IF{unlocking was successful} 
    \STATE \code{continue}
  \ELSE 
    \STATE move $init$ back and \code{break}
  \ENDIF
\ENDFOR
\STATE in $W$ create default constructor and constructor for initializing all fields
\IF{all $inits$ were moved \AND initializer evaluation is order independent}
  \STATE change the constructor call for $w$ to initialize the fields and remove $inits$
\ELSE
  \STATE merge consecutive $inits$ to common initializer blocks
\ENDIF

%\FORALL{$f\in fs$}
%  \STATE \assert $f$ is not static
%  \FORALL{uses $v$ of $f$}
%    \STATE qualify $v$ with a locked access to $w$
%  \ENDFOR
%  \STATE remove $f$
%  \STATE $\refactoring{Insert Field}(W, f)$
%  \IF{$f$ has initialiser}
%    \STATE lock flow dependencies of $f$
%    \STATE $e \leftarrow \text{initialiser of $f$}$
%    \STATE remove initialiser of $f$
%    \STATE add $e$ as argument to initialisation of $w$
%    \STATE $p \leftarrow \text{new parameter of same name and type as $f$}$
%    \FORALL{constructors $cd$ of $W$}
%      \STATE add copy of $p$ as parameter of $W$
%      \STATE add assignment from parameter to $f$ to body of $cd$
%    \ENDFOR
%  \ENDIF
%\ENDFOR
\end{algorithmic}
\end{algorithm}

This is only a bare-bones specification. The implementation additionally allows to encapsulate the extracted fields, and to move the wrapper class $W$ to the toplevel.

\subsection{\refactoring{Extract Constant}}
This refactoring extracts a constant expression into a field. Implemented in \sourcelink{ExtractTemp/ExtractConstant.jrag}.

\begin{algorithm}
\caption{$\refactoring{Extract Constant}(e : \type{Expr}, n : \type{Name})$}
\begin{algorithmic}[1]
\REQUIRE Java
\ENSURE Java $\cup$ locked dependencies
\medskip
\STATE \assert $\util{extractible}(e)$
\STATE $h \leftarrow \util{enclosingType}(e)$
\STATE $\util{lock}(e)$
\STATE $f \leftarrow \type{Field}([\text{\code{static}}; \text{\code{final}}; \text{\code{public}}], \locked{\util{effectiveType}(e)}, e)$
\STATE $\util{replaceExpr}(e, \locked{f})$
\STATE $\util{insertField}(h, f)$
\end{algorithmic}
\end{algorithm}

We first ensure that $e$ is extractible: this means that its type cannot be \code{void}, and it cannot be a reference to a type or package, nor can it be the keyword \code{super}; furthermore, it cannot be on the right-hand side of a dot.

Then all dependencies within $e$ are locked, and we construct a \code{public} \code{static} \code{final} field $f$ that is initialised to $e$. The type of $f$ is the \emph{effective type} of $e$, which is the same as the type of $e$, except when the type of $e$ is an anonymous class, in which case the effective type is its superclass, or when the type of $e$ is a captured type variable, in which case the effective type is its upper bound.

Now $e$ is simply replaced by a locked access to $f$, and $f$ is inserted into the enclosing type. 

\subsection{\refactoring{Extract Method}}
See ECOOP 2009 publication. (TODO)

\subsection{\refactoring{Extract Temp}}
This refactoring extracts an expression into a local variable. Implemented in \sourcelink{ExtractTemp/ExtractTemp.jrag}.

\begin{algorithm}
\caption{$\refactoring{Extract Temp}(e : \type{Expr}, n : \type{Name})$}
\begin{algorithmic}[1]
\REQUIRE Java
\ENSURE Java
\medskip
\STATE $v \leftarrow [\refactoring{Insert Local Variable}](\util{enclosingStmt}(e), \util{effectiveType}(e), n)$
\STATE $[\refactoring{Extract Assignment}](v, e)$
\STATE $\refactoring{Merge Declaration}(v)$
\end{algorithmic}
\end{algorithm}

We first perform the \refactoring{Insert Local Variable} refactoring to create a local variable $v$ with the same type as $e$ and with name $e$ in the enclosing body declaration. Then we use \refactoring{Extract Assignment} to extract $e$ into an assignment $a$ to $v$. Finally, we merge $a$ with the declaration $v$, turning it into its initialiser if possible.

\subsubsection{\refactoring{Insert Local Variable}}
The refactoring inserts a local variable before a given statement. Implemented in \sourcelink{ExtractTemp/IntroduceUnusedLocal.jrag}.

\begin{algorithm}
\caption{$\refactoring{Insert Local Variable}(s : \type{Stmt}, t : \type{Type}, n : \type{Name}) : \type{LocalVarDecl}$}
\begin{algorithmic}[1]
\REQUIRE Java
\ENSURE Java $\cup$ locked names
\medskip
\STATE $b \leftarrow \util{enclosingBlock}(s)$
\STATE \assert $\util{canIntroduceLocal}(b, n)$
\STATE $\util{lockNames}(b, n)$
\STATE $v \leftarrow \type{LocalVarDecl}(\locked{t}, n)$
\STATE $\util{insertStmtBefore}(s, v)$
\RETURN $v$
\end{algorithmic}
\end{algorithm}

The refactoring ensures that a variable of name $n$ can be introduced into the enclosing block $b$. This is not possible, for instance, if there already is a local variable of the same name in an enclosing scope. Then all references to variables of name $n$ are locked within $b$, and the local variable declaration is constructed and inserted into $b$.

\subsubsection{\refactoring{Extract Assignment}}
This refactoring extracts an expression into an assignment to a local variable. Implemented in \sourcelink{ExtractTemp/ExtractAssignment.jrag}.

\begin{algorithm}
\caption{$\refactoring{Extract Assignment}(v : \type{LocalVarDecl}, e : \type{Expr}) : \type{Assignment}$}
\begin{algorithmic}[1]
\REQUIRE Java
\ENSURE Java $\cup$ locked dependencies
\medskip
\STATE \assert $\util{extractible}(e)$
\STATE $a \leftarrow \type{Assignment}(\locked{v}, e)$
\IF{$\util{inExprStmt}(e)$}
  \STATE $\util{replaceExpr}(e, a)$
\ELSE
  \STATE $s \leftarrow \util{enclosingStmt}(e)$
  \STATE $\util{lock}(e)$
  \STATE $\util{insertStmtBefore}(s, a)$
  \STATE $\util{replaceExpr}(e, \locked{v})$
\ENDIF
\RETURN $a$
\end{algorithmic}
\end{algorithm}

The refactoring ensures that $e$ is an extractible expression and constructs the assignment $a$. If $e$ is in an expression statement, we can directly replace it with $a$. Otherwise, we insert it before the enclosing statement, locking dependencies in $e$ and replacing it by a variable access.

\subsubsection{\refactoring{Merge Variable Declaration}}
This refactoring merges a variable declaration with the assignment immediately following it, if that assignment is an assignment to the same variable. Implemented in \sourcelink{ExtractTemp/MergeVarDecl.jrag}.

\begin{algorithm}
\caption{$\refactoring{Merge Variable Declaration}(v : \type{LocalVarDecl})$}
\begin{algorithmic}[1]
\REQUIRE Java $\setminus$ multi-declarations
\ENSURE Java
\medskip
\IF{$\util{hasInit}(v)$}
  \RETURN
\ENDIF
\STATE $s \leftarrow \util{followingStmt}(v)$
\IF{$\util{isAssignmentTo}(s, v)$}
  \STATE $\util{setInit}(v, \util{rhs}(s))$
  \STATE $\util{removeStmt}(s)$
\ENDIF
\end{algorithmic}
\end{algorithm}

\subsection{\refactoring{Inline Constant}}
This refactoring inlines a constant field into all its uses. Implemented in \sourcelink{InlineTemp/InlineConstant.jrag}.

\begin{algorithm}
\caption{$\refactoring{Inline Constant}(f : \type{Field})$}
\begin{algorithmic}[1]
\REQUIRE Java $\setminus$ implicit assignment conversion
\ENSURE Java
\medskip
\FORALL{uses $u$ of $f$}
  \STATE $\refactoring{Inline Constant}(u)$
\ENDFOR
\STATE $\refactoring{Remove Field}(f)$
\end{algorithmic}
\end{algorithm}

\begin{algorithm}
\caption{$\refactoring{Inline Constant}(u : \type{FieldAccess})$}
\begin{algorithmic}[1]
\REQUIRE Java
\ENSURE Java $\cup$ locked dependencies
\medskip
\STATE $f \leftarrow \text{field accessed by $u$}$
\STATE \assert $f$ is \code{final} and \code{static}, and has an initialiser
\STATE $e \leftarrow \text{locked copy of the initialiser of $f$}$
\STATE \assert if $u$ is qualified, then its qualifier is a pure expression
\STATE replace $u$ with $e$, discarding its qualifier if any
\end{algorithmic}
\end{algorithm}

\begin{algorithm}
\caption{$\refactoring{Remove Field}(f : \type{Field})$}
\begin{algorithmic}[1]
\REQUIRE Java
\ENSURE Java
\medskip
\IF{$f$ is not used and if it has an initialiser, it is pure}
  \STATE remove $f$
\ENDIF
\end{algorithmic}
\end{algorithm}

\subsection{\refactoring{Inline Method}}
% TODO:
This refactoring is inverse of $\refactoring{Extract Method}$. Implemented in \sourcelink{InlineMethod/}; 
see Algorithms~\ref{alg:InlineMethod}, \ref{alg:InlineMethodAccess}, \ref{alg:InlineToAnonymousMethod}, %
\ref{alg:IntroduceOutParameter}, %
\ref{alg:OpenVariables}, \ref{alg:InlineAnonymousMethod}, \ref{alg:InlineBlock}.


\begin{algorithm}[p]
\caption{$\refactoring{Inline Method}(m \colon \type{Method})$}
\label{alg:InlineMethod}
\begin{algorithmic}[1]
\REQUIRE Java
\ENSURE Java $\cup$ fresh variables, \code{with} statement, locked names
\medskip
  \FORALL{$methosAccess$ in $\util{polyUses}(m)$} % TODO: polyUses
    \STATE $\refactoring{Inline Method Access}(methodAccess)$
  \ENDFOR
  \STATE $\refactoring{Remove Method}(m)$ \orelse\ $\refactoring{Id}()$
\end{algorithmic}
\end{algorithm}


\begin{algorithm}[p]
\caption{$\refactoring{Inline Method Access}(ma \colon \type{MethodAccess})$}
\label{alg:InlineMethodAccess}
\begin{algorithmic}[1]
\REQUIRE Java
\ENSURE Java $\cup$ fresh variables, \code{with} statement, locked names
\medskip
  \STATE $am \leftarrow \refactoring{Inline To Anonymous Method}(ma)$
  \STATE $\refactoring{Introduce Out Parameter}(am)$
  \STATE $\refactoring{Open Variables}(am)$
  \STATE $node \leftarrow \refactoring{Inline Anonymous Method}(am)$
  \IF{$node$ is a $\type{Block}$}
    \STATE $\refactoring{Inline Block}(node)$
  \ENDIF
\end{algorithmic}
\end{algorithm}


\begin{algorithm}[p]
\caption{$\refactoring{Inline To Anonymous Method}(am \colon \type{MethodAccess}) : \type{AnonymousMethod}$}
\label{alg:InlineToAnonymousMethod}
\begin{algorithmic}[1]
\REQUIRE Java
\ENSURE Java $\cup$ \code{with} statement, locked names
\medskip
  \STATE \assert $\util{target}(ma)$ is unambiguous
  \STATE $target \leftarrow \util{target}(ma)$
  \STATE \assert $target$ has a body
  \STATE $am \leftarrow$ copy target as anonymous method, with locked names, 
  			unfolded synchronize and arguments from $ma$
  \IF{$ma$ is right child of $\type{Dot}$}
    \STATE add \code{with} statement around the body of $am$ \\ mapping \code{this} to qualifier of $ma$
    \STATE replace qualifier and the access with $am$
  \ELSE
    \STATE replace $ma$ with $am$
  \ENDIF
  \RETURN $am$
\end{algorithmic}
\end{algorithm}

\begin{algorithm}[p]
\caption{$\refactoring{Introduce Out Parameter}(am \colon \type{AnonymousMethod})$}
\label{alg:IntroduceOutParameter}
\begin{algorithmic}[1]
\REQUIRE
\ENSURE adds fresh variables
\medskip
  \STATE $\util{eliminateVarargs}()$
  \STATE \dots
\end{algorithmic}
\end{algorithm}


\begin{algorithm}[p]
\caption{$\refactoring{Open Variables}(am \colon \type{AnonymousMethod})$}
\label{alg:OpenVariables}
\begin{algorithmic}[1]
\REQUIRE ??? Java
\ENSURE ??? Java $\cup$ fresh variables, \code{with} statement, locked names
\medskip
  \STATE
\end{algorithmic}
\end{algorithm}


\begin{algorithm}[p]
\caption{$\refactoring{Inline Anonymous Method}(am \colon \type{AnonymousMethod}) : ASTNode$}
\label{alg:InlineAnonymousMethod}
\begin{algorithmic}[1]
\REQUIRE ??? Java
\ENSURE ??? Java $\cup$ fresh variables, \code{with} statement, locked names
\medskip
  \STATE
\end{algorithmic}
\end{algorithm}


\begin{algorithm}[p]
\caption{$\refactoring{Inline Block}(b \colon \type{Block})$}
\label{alg:InlineBlock}
\begin{algorithmic}[1]
\REQUIRE ??? Java
\ENSURE ??? Java $\cup$ fresh variables, \code{with} statement, locked names
\medskip
  \STATE
\end{algorithmic}
\end{algorithm}



\subsection{\refactoring{Inline Temp}}
This refactoring inlines a local variable into all its uses. Implemented in \sourcelink{InlineTemp/InlineTemp.jrag}.

\begin{algorithm}
\caption{$\refactoring{Inline Temp}(d : \type{LocalVarDecl})$}
\begin{algorithmic}[1]
\REQUIRE Java
\ENSURE Java
\medskip
\STATE $a = \lfloor\refactoring{Split Declaration}\rfloor(d)$
\STATE $\lfloor\refactoring{Inline Assignment}\rfloor(a)$
\STATE $\lfloor\refactoring{Remove Decl}\rfloor(v)$
\end{algorithmic}
\end{algorithm}

\begin{algorithm}
\caption{$\refactoring{Split Declaration}(d : \type{LocalVarDecl}) : \type{Assignment}$}
\begin{algorithmic}[1]
\REQUIRE Java $\setminus$ multi-declarations
\ENSURE Java $\cup$ locked names, first-class array init
\medskip
\STATE $a = \type{Assignment}(\locked{d}, \util{getInit}(d))$  
\STATE $\util{insertStmtAfter}(d, a)$
\STATE $\util{removeInit}(d)$
\RETURN $a$
\end{algorithmic}
\end{algorithm}

\begin{algorithm}
\caption{$\refactoring{Inline Assignment}(a : \type{Assignment})$}
\begin{algorithmic}[1]
\REQUIRE Java
\ENSURE Java $\cup$ locked dependencies
\medskip
\STATE $x = \util{lhs}(a)$
\STATE \assert $x$ is local variable
\STATE $e = \util{rhs}(a)$
\STATE $\util{lock}(e)$
\STATE $U = \{ u \mid u\rightarrow_r x\}$
\FORALL{$u\in U$}
  \STATE \assert $\neg\exists x'.u\rightarrow_r x' \wedge x\neq x'$
  \STATE \assert $u$ is not an lvalue
  \STATE \assert $u,x$ are in same body declaration
  \STATE $\util{replaceExpr}(u, \util{copy}(e))$
\ENDFOR
\IF{$U\neq\emptyset$}
  \STATE $\util{removeStmt}(a)$
\ENDIF
\end{algorithmic}
\end{algorithm}

\begin{algorithm}
\caption{$\refactoring{Remove Decl}(d : \type{LocalVarDecl})$}
\begin{algorithmic}[1]
\REQUIRE Java $\setminus$ multi-declarations
\ENSURE Java
\medskip
\IF{$\neg\util{hasInit}(d)\wedge\neg\exists u.u\rightarrow_b d$}
  \STATE $\util{removeStmt}(d)$
\ENDIF
\end{algorithmic}
\end{algorithm}

\subsection{\refactoring{Insert Method}}
This refactoring inserts a method into a type declaration. Implemented in \sourcelink{Move/InsertUnusedMethod.jrag}; see Algorithms~\ref{alg:InsertMethod},~\ref{alg:canIntroduceMethod},~\ref{alg:typesToMakeAbstract}.

\begin{algorithm}[p]
\caption{$\refactoring{Insert Method}(m : \type{Method}, T : \type{Type})$}\label{alg:InsertMethod}
\begin{algorithmic}[1]
\REQUIRE Java
\ENSURE Java $\cup$ locked method names
\medskip
  \STATE $\util{lockMethodNames}(\util{name}(m))$
  \STATE \assert $\util{canIntroduceMethod}(m, T)$
  \STATE \assert \NOT $\util{isDynamicallyCallable}(m)$ 
  \STATE \assert $\{\util{name}(td) | \type{TypeDecl}\ td\in\util{below}(m)\} 
  		$\\$\qquad\qquad\cap\  
		\{\util{name}(t) | t\text{ is enclosing type of }T \vee t = T\} = \emptyset$
  \STATE insert method $m$ into the type $T$
  \IF{$m$ is \code{abstract}}
     \FORALL{$type$ in $\util{typesToMakeAbstract}(m)$}
       \STATE $\refactoring{Make Type Abstract}(type)$
     \ENDFOR
  \ENDIF
\end{algorithmic}
\end{algorithm}


\begin{algorithm}[p]
\caption{$\util{canIntroduceMethod}(m : \type{Method}, T : \type{Type})$}\label{alg:canIntroduceMethod}
\begin{algorithmic}[1]
  \STATE \assert $m$ is not \code{static} \OR $T$ is not inner
  \STATE \assert there is no local method in $T$ with same signature errasure as $m$
  \STATE \assert if there are any like-named methods in superclasses, we must be able to override or hide them, and
  			similarly for subclasses
\end{algorithmic}
\end{algorithm}

\begin{algorithm}[p]
\caption{$\util{typesToMakeAbstract}(m : \type{Method}) : \settp{Type}$}\label{alg:typesToMakeAbstract}
\begin{algorithmic}[1]
  \STATE do DFS from $\util{hostType}(m)$ through child types \\
    but do not visit a type that declares a method that \emph{overrides} $m$ \\
    (in particular, visit a type in a different package, even if it can't override $m$)
  \RETURN set of all visited types
\end{algorithmic}
\end{algorithm}


\subsection{\refactoring{Introduce Factory}}
This refactoring introduces a static factory method as a replacement for a given constructor, and updates all uses of the constructor to use this method instead. Implemented in \sourcelink{IntroduceFactory/IntroduceFactory.jrag}.

\begin{algorithm}
\caption{$\refactoring{Introduce Factory}(cd : \type{ConstructorDecl})$}
\begin{algorithmic}[1]
\REQUIRE Java
\ENSURE Java $\cup$ locked names
\medskip
\STATE $f \leftarrow \util{createFactoryMethod}(cd)$
\FORALL{$c \in \{c : \type{ClassInstanceExpr} \mid \util{original}(\util{decl}(c)) = cd \wedge \neg\util{hasTypeDecl}(c)$}
  \STATE $\util{replaceExpr}(c, \type{MethodCall}(\locked{f}, \util{getArgs}(c)))$
\ENDFOR
\end{algorithmic}
\end{algorithm}

We first use \util{createFactoryMethod} to create the factory method corresponding to constructor $cd$ and insert it into the host type of $cd$. The factory method has the same signature as $cd$, but it has its own copies of all type variables of the host type used in $cd$.

Then every class instance expression that uses $cd$ or a parameterised instance of $cd$ and does not have its own anonymous type declaration is replaced by a call to the factory method.

\subsection{\refactoring{Introduce Indirection}}
This refactoring creates a static method $m'$ in type $B$ that delegates to a method $m$ in type $A$. Implemented in \sourcelink{IntroduceIndirection/IntroduceIndirection.jrag}; see Algorithm~\ref{alg:IntroduceIndirection}.

\begin{algorithm}[p]
\caption{$\refactoring{Introduce Indirection}(m : \type{Method}, B : \type{ClassOrInterface})$}
\label{alg:IntroduceIndirection}
\begin{algorithmic}[1]
\REQUIRE Java
\ENSURE Java $\cup$ locked names, \code{return void}
\medskip
\STATE \assert $B$ is non-library
\STATE $fn \leftarrow \text{fresh method name}$
\STATE $m' \leftarrow \text{copy of $m$ with locked names and empty body}$
\STATE set name of $m'$ to $fn$
\STATE $xs \leftarrow \text{locked accesses to parameters of $m'$}$
\STATE set body of $m'$ to \code{return}\xspace$m$\code{(}$xs$\code{);}
\STATE $\refactoring{Insert Method}(\util{hostType}(m), m')$
\STATE $\refactoring{Make Method Static}(m')$
\STATE $\refactoring{Move Static Method}(m', B)$
\end{algorithmic}
\end{algorithm}

\subsection{\refactoring{Introduce Parameter}}
This refactoring turns an expression into a parameter of the surrounding method. Implemented in \sourcelink{ChangeMethodSignature/IntroduceParameter.jrag}; see Algorithm~\ref{alg:IntroduceParameter}.

\begin{algorithm}[p]
\caption{$\refactoring{Introduce Parameter}(e : \type{Expr}, n : \type{Name})$}
\label{alg:IntroduceParameter}
\begin{algorithmic}[1]
\REQUIRE Java
\ENSURE Java $\cup$ locked names
\medskip
\STATE \assert $n$ is a valid name
\STATE \assert $e$ is extractible and constant
\STATE \assert $e$ appears within a method $m$
\STATE \assert $m$ is not overridden by and does not override any other methods
\STATE \assert $m$ has no parameter or local variable $n$
\STATE $\util{lockMethodCalls}(\util{name}(m))$
\STATE $t \leftarrow \text{effective type of $e$}$
\STATE $p \leftarrow \text{new parameter of type $t$ and name $n$}$
\STATE insert $p$ as the first parameter of $m$
\STATE replace $e$ with locked access to $p$
\FORALL{calls $c$ to $m$}
  \STATE insert a locked copy of $e$ as first argument of $c$
\ENDFOR
\end{algorithmic}
\end{algorithm}

\subsection{\refactoring{Introduce Parameter Object}}

\subsection{\refactoring{Make Method Static}}
This refactoring makes a method static. Implemented in \sourcelink{MakeMethodStatic/MakeMethodStatic.jrag}; see Algorithm~\ref{alg:MakeMethodStatic}.

\begin{algorithm}[p]
\caption{$\refactoring{Make Method Static}(m : \type{Method})$}\label{alg:MakeMethodStatic}
\begin{algorithmic}[1]
\REQUIRE Java
\ENSURE Java $\cup$ return \code{void}, fresh variables, \code{with} statement, locked names, demand \code{final} modifier
\medskip
  \STATE \assert $m$ has a body
  \STATE $newMethod\leftarrow \util{copy}(m)$
  \STATE $delegator\leftarrow m$
  \STATE $\util{lockMethodNames}(\util{name}(delegator))$
  \STATE add \code{static} modifier to $newMethod$
  \STATE add new parameter to $newMethod$ with fresh name, type locked to $\util{hostType}(m)$, and demand final
  \STATE put a \code{with} statement around the body of $newMethod$ mapping \code{this} to the new parameter
  \STATE $\refactoring{Close Over Variables}(newMethod)$
  \STATE change the block of $delegator$ method to a call to $newMethod$ \\
  	with \code{this} and parameters of $delegator$ as arguments
  \STATE $\refactoring{Insert Method}(\util{hostType}(delegator), newMethod)$
\end{algorithmic}
\end{algorithm}



\subsection{\refactoring{Move Inner To Toplevel}}

\begin{algorithm}
\caption{$\refactoring{Move Inner to Toplevel}(M : \type{MemberType})$}
\label{alg:MoveMemberTypeToToplevel}
\begin{algorithmic}[1]
\REQUIRE Java $\setminus$ implicit constructor invocations
\ENSURE Java $\cup$ \code{with}, locked names
\medskip
\STATE $\util{lockTypeNames}(\util{programRoot}(), \util{name}(M))$
\STATE $\util{lockNames}(M)$
\IF{$\neg\util{isStatic}(M)$}
  \STATE $[A_n;\ldots;A_1] = \util{enclosingTypes}(M)$
  \STATE $[f_n;\ldots;f_1] = [\type{Field}(\locked{A_n}); \ldots; \type{Field}(\locked{A_1})]$
  \FORALL{$i\in\{1,\ldots,n\}$}
    \STATE $\util{addField}(M, f_i)$
    \FORALL{$c\in\util{constructors}(M)$}
      \STATE $p := \type{Parameter}(\locked{A_i})$
      \STATE $\util{addParameter}(c, p)$
      \IF{$\util{isChaining}(c)$}
        \STATE $\util{addArgument}(\util{thisCall}(c), \locked{p})$
      \ELSE
        \STATE $\util{addStmt}(c, \type{Assignment}(\locked{f_i}, \locked{p}))$
      \ENDIF
    \ENDFOR
  \ENDFOR
  \FORALL{$c\in\util{constructors}(M)$}
    \FORALL{$u\in\util{nonChainingInvocations}(c)$}
      \STATE $es = \util{enclosingInstances}(u)$
      \STATE \assert $|es|=n$
      \STATE $\util{addArguments}(u, es)$
      \STATE $\util{discardQualifier}(u)$
    \ENDFOR
  \ENDFOR
  \FORALL{$m\in\util{callables}(M)$}
    \IF{$\util{hasBody}(m)$}
      \STATE $b := \util{getBody}(m)$
      \STATE $b' := \type{With}([\locked{f_n}; \ldots; \locked{f_1}; \text{\code{this}}], b)$
      \STATE $\util{setBody}(m, b')$
    \ENDIF
  \ENDFOR
\ENDIF
\STATE $\util{removeBodyDecl}(M)$
\STATE $\util{addToplevelType}(M)$
\end{algorithmic}
\end{algorithm}

\subsection{\refactoring{Move Instance Method}}
See WRT 2009 publication.

\subsection{\refactoring{Move Members}}
In order to move Field, static methods, and member types, we simply lock all references to them, as well as all names contained in them, and (for fields) the flow dependencies of their initialiser, and then move them inside the AST.

\subsection{\refactoring{Promote Temp to Field}}

\subsection{\refactoring{Pull Up}}
This refactoring pulls up a method $m$ from its host class $B$ to the super class $A$. Implemented in \sourcelink{PullUp/PullUpMethod.jrag}; see Algorithm~\ref{alg:PullUpMethod}.

\begin{algorithm}[p]
\caption{$\refactoring{Pull Up Method}(m : \type{Method})$}\label{alg:PullUpMethod}
\begin{algorithmic}[1]
\REQUIRE Java
\ENSURE Java $\cup$ locked names
\medskip
\STATE \assert the host type of $m$ $B$ is a non-library class
\STATE \assert the superclass $A$ of $B$ is also non-library
\STATE $m' \leftarrow \text{copy of $m$ with locked names}$
\STATE translate type variables in $m'$ from $B$ to $A$
\STATE $\refactoring{Insert Method}(A, m')$
\STATE remove $m$ from $B$
\end{algorithmic}
\end{algorithm}

TODO: explain translation of type variables; this is basically a right-inverse of the type variable substitution that happens when inheriting a method

Note that \refactoring{Insert Method} ensures that the inserted method is not called from anywhere.

\subsection{\refactoring{Push Down}}
This refactoring pushes a method down to all subclasses of its defining class. Implemented in \sourcelink{PushDown/PushDownMethod.jrag}; see Algorithms~\ref{alg:TriviallyOverride},~\ref{alg:RemoveMethod},~\ref{alg:MakeMethodAbstract},~\ref{alg:MakeTypeAbstract},~\ref{alg:PushDownVirtualMethod}.

Types that inherit a method $m$ include the host type of $m$.

\begin{algorithm}[p]
\caption{$\refactoring{Trivially Override}(B : \type{Type}, m : \type{VirtualMethod}) : \option{\type{MethodCall}}$}
\label{alg:TriviallyOverride}
\begin{algorithmic}[1]
\REQUIRE Java $\setminus$ implicit method modifiers
\ENSURE Java $+$ locked names, \code{return void}
\medskip
\STATE \assert $m$ is not \code{final}
\IF{$m$ not a member method of $B$}
  \RETURN \None
\ENDIF
\STATE $m' \leftarrow \text{copy of $m$ with locked names}$
\IF{$m$ is \code{abstract}}
  \STATE $\refactoring{Insert Method}(B, m')$
  \RETURN \None
\ELSE
  \STATE $xs \leftarrow \text{list of locked accesses to parameters of $m'$}$
  \STATE $c \leftarrow \text{\code{super.}$m$\code{(}$xs$\code{)}}$
  \STATE set body of $m'$ to \code{return}\xspace $c$\code{;}
  \STATE $\refactoring{Insert Method}(B, m')$
  \RETURN \Some{c}
\ENDIF
\end{algorithmic}
\end{algorithm}

\begin{algorithm}[p]
\caption{$\refactoring{Remove Method}(m : \type{Method})$}
\label{alg:RemoveMethod}
\begin{algorithmic}[1]
\REQUIRE Java
\ENSURE Java
\medskip
\STATE \assert $(\util{uses}(m)\cup\util{calls}(m))\setminus\util{below}(m)=\emptyset$
\STATE $o \leftarrow \{ m' \mid m <: m' \}$
\IF{$o\neq\emptyset\wedge\forall m'\in o.\text{$m'$ is abstract}$}
  \FORALL{$B$ in $\util{typesToMakeAbstract}(m)$}
    \STATE $\refactoring{Make Type Abstract}(B)$
  \ENDFOR
\ENDIF
\STATE remove $m$
\end{algorithmic}
\end{algorithm}

\begin{algorithm}[p]
\caption{$\refactoring{Make Method Abstract}(m : \type{Method})$}
\label{alg:MakeMethodAbstract}
\begin{algorithmic}[1]
\REQUIRE Java
\ENSURE Java
\medskip
\STATE \assert $m$ is not \code{native}, \code{static}, \code{private}, nor \code{final}
\STATE \assert there are no static calls to $m$ (e.g., \code{super}-call)
\FORALL{$B$ in $\util{typesToMakeAbstract}(m)$}
  \STATE $\refactoring{Make Type Abstract}(B)$
\ENDFOR
\STATE make $m$ \code{abstract}
\end{algorithmic}
\end{algorithm}

\begin{algorithm}[p]
\caption{$\refactoring{Make Type Abstract}(T : \type{Type})$}
\label{alg:MakeTypeAbstract}
\begin{algorithmic}[1]
\REQUIRE Java
\ENSURE Java
\medskip
\IF{$T$ is interface}
    \RETURN
\ENDIF
\STATE \assert $T$ is class and never instantiated
\STATE make $T$ \code{abstract}
\end{algorithmic}
\end{algorithm}

\begin{algorithm}[p]
\caption{$\refactoring{Push Down Virtual Method}(m : \type{VirtualMethod})$}
\label{alg:PushDownVirtualMethod}
\begin{algorithmic}[1]
\REQUIRE Java
\ENSURE Java $\cup$ locked names
\medskip
\FORALL{types $B<:\util{hostType}(m)$}
  \STATE $c \leftarrow \refactoringNoExt{Trivially Override}(B, m)$
  \IF{$c\neq\util{None}$}
    \STATE $\refactoring{Inline Method}(c)$
  \ENDIF
\ENDFOR
\STATE $\refactoring{Remove Method}(m)$
\STATE \qquad\orelse$\refactoring{Make Method Abstract}(m)$
\STATE \qquad\orelse$\refactoring{Id}()$
\end{algorithmic}
\end{algorithm}

\subsection{\refactoring{Rename}}
This family of refactorings is used for renaming named program entities. Implemented in \sourcelink{Renaming/}.

\begin{algorithm}
\caption{$\refactoring{Rename Field}(f : \type{Field}, n : \type{Name})$}
\begin{algorithmic}[1]
\REQUIRE Java
\ENSURE Java $\cup$ locked names
\medskip
\STATE \assert $n$ is a valid name
\STATE \assert host type of $f$ contains no other field of name $n$
\STATE $\util{lockNames}(\{n, \util{name}(f)\})$
\STATE set name of $f$ to $n$
\end{algorithmic}
\end{algorithm}

Refactoring \refactoring{Rename Field} changes the name of a field $f$ to $n$. It ensures that $n$ is indeed a valid name and that the host type of $f$ contains no other field called $n$. It then globally locks all accesses to variables, types, or packages named either $n$ or $\util{name}(f)$, and changes the name of $f$ to $n$.

\begin{algorithm}
\caption{$\refactoring{Rename Local}(v : \type{Local}, n : \type{Name})$}
\begin{algorithmic}[1]
\REQUIRE Java
\ENSURE Java $\cup$ locked names
\medskip
\STATE \assert $n$ is a valid name
\STATE \assert enclosing block of $v$ is neither contained in, nor contains the scope of another local named $n$
\STATE $\util{lockNames}(\util{block}(v), \{n, \util{name}(f)\})$
\STATE set name of $v$ to $n$
\end{algorithmic}
\end{algorithm}

Refactoring \refactoring{Rename Local} changes the name of a local variable or parameter $v$ to $n$. It ensures that $n$ is indeed a valid name and that the renaming $v$ to $n$ will not violate the rule that scopes of local variables of the same name cannot be nested. It then again locks all accesses to variables, types, or packages named either $n$ or $\util{name}(v)$, but only within the enclosing block of $v$, and changes the name of $v$ to $n$.

\subsection{\refactoring{Self-Encapsulate Field}}
This refactoring makes a field private, rerouting all accesses to it through getter and setter methods. Implemented in \sourcelink{SelfEncapsulateField/SelfEncapsulateField.jrag}; see Algorithm~\ref{alg:Self-EncapsulateField}.

\begin{algorithm}[p]
\caption{$\refactoring{Self-Encapsulate Field}(f : \type{Field})$}\label{alg:Self-EncapsulateField}
\begin{algorithmic}[1]
\REQUIRE Java $\setminus$ abbreviated assignments
\ENSURE Java $\cup$ locked names
\medskip
\STATE create getter method $g$ for $f$
\STATE if $f$ is not \code{final}, create setter method $s$ for it
\FORALL{all uses $u$ of $f$ and its substituted copies}
  \IF{$u\not\in\util{below}(g)\cup\util{below}(s)$}
    \IF{$u$ is an rvalue}
      \STATE replace $u$ with locked access to $g$
    \ELSE
      \IF{$f$ is not \code{final}}
        \STATE $q \leftarrow \text{qualifier of $u$, if any}$
        \STATE $r \leftarrow \text{RHS of assignment for which $u$ is LHS}$
        \STATE replace $u$ with locked access to $s$ on argument $r$, qualified with $q$ if applicable
      \ENDIF
    \ENDIF
  \ENDIF
\ENDFOR
\end{algorithmic}
\end{algorithm}

By ``abbreviated assignment'' we mean \code{x+=y} and friends, as well as increment and decrement expressions. The language restriction tries to expand these into normal assignments, but may fail if the data flow is too complicated. If it succeeds, every lvalue will appear on the left hand side of a (simple) assignment.

Note that even when $f$ is final there may still be assignments to $f$ from within constructors; we cannot encapsulate these assignments, so we skip them.


\clearpage

\section{Node Types}
See Fig.~\ref{fig:node types}. We also use the non-node type \type{Name} to represent names.

\begin{figure}[hb]
\begin{center}
\begin{tabular}{|l|p{5cm}|}
\hline
\textbf{Node Type} & \textbf{Description} \\ \hline\hline
\type{ClassOrInterface} & either a class or an interface; is a \type{Type} \\
\type{Field} & field declaration \\
\type{LocalVar} & local variable declaration \\
\type{MemberType} & type declared inside another type; is a \type{Type} \\
\type{Method} & method declaration \\
\type{MethodCall} & method call \\
\type{Package} & package \\
\type{Type} & type declaration \\
\type{VirtualMethod} & non-\code{private} instance method; is a \type{Method} \\
\hline
\end{tabular}
\end{center}
\caption{Node Types}
\label{fig:node types}
\end{figure}



\section{Utility Functions}
See Fig.~\ref{fig:utility}.

\begin{figure}[hb]
\begin{center}
\begin{tabular}{|l|p{5.3cm}|}
\hline
\textbf{Name} & \textbf{Description} \\ \hline\hline
$\util{below}(n)$ & returns the set of all nodes below $n$ in the syntax tree \\
$\util{calls}(m)$ & returns all calls that may dynamically resolve to method $m$; can be a conservative over-approximation \\
$\util{hostPkg}(e)$ & returns the package of the compilation unit containing $e$ \\
$\util{hostType}(e)$ & returns the closest enclosing type declaration around $e$ \\
$\util{lockMethodCalls}(n)$ & locks all calls to methods named $n$ anywhere in the program \\
$\util{lockNames}(n)$ & locks all names anywhere in the program that refer to a declaration with name $n$ \\
$\util{name}(e)$ & returns the name of program entity $e$ \\
$\util{uses}(m)$ & returns all calls that statically bind to method $m$ \\
\hline
\end{tabular}
\end{center}
\caption{Utility Functions}
\label{fig:utility}
\end{figure}


\clearpage

\listofalgorithms
\end{document}
