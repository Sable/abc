This provides a description of the JastAdd module system.

\subsection{Module System Overview}

A module consists of aspects and classes, and defines the set of external
modules that are visible to the module's members. The module system is based
on the object-oriented java module system presented in \cite{modulesastypes}

\subsection{Declaration, Membership and Exports}

A module is defined in a \texttt{.module} file, and is headed by the module's name:

\begin{lstlisting}[caption={Module Declaration}]
//file prettyprinter.module
module prettyprinter;
...
\end{lstlisting}

Membership to a module is defined in the compilation unit of the members,
similar to a package declaration:

\begin{lstlisting}[caption={Module Membership}]
//file PrettyPrinter.jrag
module prettyprinter; //module membership
public aspect PrettyPrinter {
	public abstract String Expr.prettyPrint();
	
	public String Add.prettyPrint() {
		return getLeft().prettyPrint() + 
				"+" + getRight().prettyPrint();
	}
	
	public String IntLit.prettyPrint() {
		return getIntLit().toString();
	}
}

//file Expr.java
module asttypes; //module membership
package expr; //package declaration
public class Expr {
...
}
\end{lstlisting}

Module and package declarations can coexist in a single compilation unit.
Modules can contain aspects and classes that span multiple packages.

Packages are also not implicitly visible outside the module unless an
\texttt{export} declaration is provided for that package. These declarations
are placed in \texttt{.module} files.

\begin{lstlisting}[caption={Export Package}]
//file asttypes.module
module asttypes;
export package expr, stmt; //export expr and stmt
...

//file prettyprinter.module
module prettyprinter;
export package *; //export all packages
...
\end{lstlisting}

As the example shows, an export package may contain a list of packages, or the
wildcard \texttt{*}, which exposes all packages in the module. Any types that
belong to a package that is not exported are not visible from outside the module.

\subsection{Imports and Instantiation}

Module definitions also contain import declarations, which specify which
other modules are visible to the members of the module. A module imports 
an \textit{instance} of another module, which allows multiple instances
of the same module to exist within a single context.

There are two ways to import a module instance: importing the singleton or
an \textbf{own} instance. Imports may also be alised to allow multiple instances
of the same module to exist in the same context, and be exported so that
they are also accessible through indirection from other modules;

\begin{lstlisting}[caption={Imports}, label={figure:imports}]
//file prettyprinter.module
module prettyprinter;
//import the singleton instance of module parserframework
import parserframework; 
//import an own instance of the module asttypes
import own asttype;
//import another instance of asttype, using an alias and export
import own asttype export as public_ast;
\end{lstlisting}

Imports of \textbf{own} instances without an explicit alias use the name
of the imported module as its alias. To be more specific, line 6 from the above
example is equivalent to
\begin{lstlisting}
import own asttype as asttype;
\end{lstlisting}

The visibility of imported modules become important for the \textbf{merge}
operation, described later.

\subsection{Lookup}

Allowing two instances of the same module to exist in the same context
requires a way to disambiguate references to the members of the instances.
We introduce \textit{module qualifiers} for type references to explicitly
choose the module to which the type belongs.

The following example uses the module definition from listing \ref{figure:imports}.
The module \texttt{prettyprinter} imports two instances of \texttt{asttype}, with the
aliases \texttt{asttype} and \texttt{public\_ast}. The aspect \texttt{PrettyPrinter},
which belongs to the module, introduces ITDs to each of the \texttt{Add} types
of the instances.

\begin{lstlisting}[caption={Module Qualfiers}]
//file PrettyPrinter.jrag
module prettyprinter;
aspect PrettyPrinter {
	...
	//modifies the Add type in asttypes
	public String asttypes::Add.prettyPrint() {
		//infix print
		return getLeft().prettyPrint() + "+" + 
				getRight().prettyPrint();
	}
	...
	//modifies the Add type in public_ast
	public String public_ast::Add.prettyPrint() {
		//postfix print
		return getLeft().prettyPrint() + 
				getRight().prettyPrint() + "+";
	}
}
\end{lstlisting}

\subsection{Merge}



\subsection{Extension}



\subsection{ITD Calculator with Modules}

\subsection{Evaluation}

