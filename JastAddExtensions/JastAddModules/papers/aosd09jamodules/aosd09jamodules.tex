% THIS IS SIGPROC-SP.TEX - VERSION 3.0
% WORKS WITH V3.1SP OF ACM_PROC_ARTICLE-SP.CLS
% JUNE 2007
%
% It is an example file showing how to use the 'acm_proc_article-sp.cls' V3.1SP
% LaTeX2e document class file for Conference Proceedings submissions.
% ----------------------------------------------------------------------------------------------------------------
% This .tex file (and associated .cls V3.1SP) *DOES NOT* produce:
%       1) The Permission Statement
%       2) The Conference (location) Info information
%       3) The Copyright Line with ACM data
%       4) Page numbering
% ---------------------------------------------------------------------------------------------------------------
% It is an example which *does* use the .bib file (from which the .bbl file
% is produced).
% REMEMBER HOWEVER: After having produced the .bbl file,
% and prior to final submission,
% you need to 'insert'  your .bbl file into your source .tex file so as to provide
% ONE 'self-contained' source file.
%
% Questions regarding SIGS should be sent to
% Adrienne Griscti ---> griscti@acm.org
%
% Questions/suggestions regarding the guidelines, .tex and .cls files, etc. to
% Gerald Murray ---> murray@acm.org
%
% For tracking purposes - this is V3.0SP - JUNE 2007

\documentclass{acm_proc_article-sp}
\usepackage{listings}
\usepackage{url}
\usepackage{cite}

\lstset{numbers=left, columns=flexible, language=[AspectJ]Java, basicstyle=\small, numberstyle=\tiny, frame=tb}
\lstset{
morekeywords={own,module,export,as,merge,{module\_interface}, replace, with, overrides, singleton, supermodule},
frame=tb,
numbers=left,
captionpos=t,
tabsize=2
}

\begin{document}

\title{Extensible Inter-type Declarations Through Modules}
%
% You need the command \numberofauthors to handle the 'placement
% and alignment' of the authors beneath the title.
%
% For aesthetic reasons, we recommend 'three authors at a time'
% i.e. three 'name/affiliation blocks' be placed beneath the title.
%
% NOTE: You are NOT restricted in how many 'rows' of
% "name/affiliations" may appear. We just ask that you restrict
% the number of 'columns' to three.
%
% Because of the available 'opening page real-estate'
% we ask you to refrain from putting more than six authors
% (two rows with three columns) beneath the article title.
% More than six makes the first-page appear very cluttered indeed.
%
% Use the \alignauthor commands to handle the names
% and affiliations for an 'aesthetic maximum' of six authors.
% Add names, affiliations, addresses for
% the seventh etc. author(s) as the argument for the
% \additionalauthors command.
% These 'additional authors' will be output/set for you
% without further effort on your part as the last section in
% the body of your article BEFORE References or any Appendices.

\numberofauthors{2} %  in this sample file, there are a *total*
% of EIGHT authors. SIX appear on the 'first-page' (for formatting
% reasons) and the remaining two appear in the \additionalauthors section.
%
\author{
% You can go ahead and credit any number of authors here,
% e.g. one 'row of three' or two rows (consisting of one row of three
% and a second row of one, two or three).
%
% The command \alignauthor (no curly braces needed) should
% precede each author name, affiliation/snail-mail address and
% e-mail address. Additionally, tag each line of
% affiliation/address with \affaddr, and tag the
% e-mail address with \email.
%
% 1st. author
\alignauthor
Neil Ongkingco\\
       \affaddr{Programming Tools Group}\\
       \affaddr{University of Oxford}\\
       \email{neil.ongkingco@gmail.com}\\
% 2nd. author
\alignauthor
Torbj\"{o}rn Ekman\\
       \affaddr{Programming Tools Group}\\
       \affaddr{University of Oxford}\\
       \email{Torbjorn.Ekman@comlab.ox.ac.uk}\\
}
% There's nothing stopping you putting the seventh, eighth, etc.
% author on the opening page (as the 'third row') but we ask,
% for aesthetic reasons that you place these 'additional authors'
% in the \additional authors block, viz.

% Just remember to make sure that the TOTAL number of authors
% is the number that will appear on the first page PLUS the
% number that will appear in the \additionalauthors section.

\maketitle
\begin{abstract}

Inter-type declarations (ITDs) in current aspect-oriented languages have
limited extensibility: it is difficult to change the behavior of
existing ITDs when extending a system written in an aspect-oriented
language. We introduce a module system that allows for ITD extensibility, and
demonstrate that modules also provide the additional benefits of 
information hiding and explicit dependency specification. The
module system is also used on a moderately-sized case study on
a Java compiler written in JastAdd, an aspect-oriented compiler
construction framework.

\end{abstract}


%A category including the fourth, optional field follows...
\category{D.3.3}{Programming Languages}{Language Constructs and Features}

\terms{Languages, Design, Module}

\keywords{Modularity, Aspect-Oriented Programming, JastAdd} 

\section{Introduction}
\label{section:introduction}

Inter-type declarations (ITDs) provide a powerful yet simple modularisation
mechanism. The possibility to extend existing classes modularly without
ahead of time planning is not only useful to separate different concerns
but also extremely convenient for modular extensibility when software
evolves.

A common critisizm of ITDs is their global scope which arguably leads to
poor information hiding and require global analysis during compilation.
Another disadvantage to more traditional extensibility patterns, e.g.,
visitors, is that the class hierarchy is destructively updated, preventing
multiple versions of a system with different sets of ITDs applied.
These problems exists to a certain extent in plain Java programs as well
and there has been a wealth of recent work on module systems to improve on
status quo. 

The emerging support for modules in Java 7 enhances information hiding and
extended module proposals such as Srniza gives hope for simulatneous
deployment of multiple versions of the same library in different modules.

import-export for information hiding -> package is too fine grained which
often leads to global scope. modules allow more precise control of
access control which leads to improved information hiding.

instantiation for multiple versions of a module (iJAM)-> 

ITDs are global in their nature which makes local
reasoning somewhat difficult, and as an extensibility mechanism they can be
improved by enabling deployment of multiple versions of libraries, 
each woven a with different set of ITDs.


Previous work show how aspects can be improved using modules for point-cut
and advice.
Aspects don't work very well without modules, due to global scope, and
implicit dependencies.
In this paper we present a module system that supports inter-type
declarations and improve their use when extending a system in a modular
fashion.

We believe that such benefits are even more important for a system with
inter-type declarations. 


%Aspect instantiation has always been a sticky subject (e.g. doubly applied
%pointcuts in AspectJ abstract aspects)


In our previous work we explain how to use ITDs as one of the main
modularisation mechanisms when building extensible compilers. Our current
work involves using the same techniques to generate IDEs for a wide range
of dialects of Java. In such an IDE we may for instance want to use numerous variants of the
same frontend that slightly differ to support different dialects. We also
want to use a pure frontend for error checking while a backend is also
needed to support code generation.
That work highlights some defiencies to ITDs from an extensibility point of
view compared to the more traditional use of visitors.

Traditionally done using visitors. Advantages of inter-type declarations:
no ahead of time planning, add state, no boiler-plate code, less error
prone since you don't have to adhere to framework conventions to enable dispatch.

Severe drawbacks from extensibility point of view:
Destructive update of the class hierarchy. The base version and the
extended version can not co-exist. This is the main motivation for our
work. Requires a global analysis for type checking, name binding, and
compilation.

We have implemented the proposed module system as an extension to the
JastAdd Extensible Java Compiler which contains support for amongst others
ITDs. 
To evaluate the module system we performed a case study where the above
mentioned extensible Java compiler built using ITDs was refactored to use
the proposed module system.
The proposed module system solves some of these problems completely in an
elegant fashion, while the burden of other problems are slightly lowered.

These are the main contributions of this paper:
\begin{itemize}
\item The design of a module system for ITDs that improves information
hiding and extensibility.
\item An implementation as a modular extension to the JastAdd Extensible
Java compiler.
\item A case study where extensible Java compiler is retrofitted to use the
proposed module system.
\end{itemize}

The rest of this paper is structured as follows: In
Section~\ref{section:itdvisitors} we present a detailed example that
highlight the merits and deficiencies of ITDs compared to a visitor based
approach. A module system that shows how that example can be improved is
presented in Section~\ref{section:jastaddmodules} and we present a case
study where an extensible Java compiler is retrofitted to use that module
system in Section~\ref{section:casestudy} where we also discuss 
the advantages it brings compared to the original impelementation. A
brief overview of the module system implementation is presented in
Section~\ref{section:implementation} and we discuss related work in
Section~\ref{section:related} and conclude in
Section~\ref{section:conclusions}.



\section{ITDs vs. Visitors}
\label{section:itdvisitors}
In this section we present two alternative implementations of a small
calculator, one using visitors and one using ITDs. It will be used as a
running example throughout the paper to illustrate the advantages and
disadvantages of the respective solution. In Section~\ref{section:jastaddmodules}
we then present a module system for ITDs that removes some of the presented
deficiencies.

\subsection{Running Example: Calculator}
Consider a tiny calculator supporting integer literals and addition, thus 
supporting composite expressions like: 1 + 2. It may seem trivial but its 
implementation illustrates some important exensibility challenges that we 
address in this paper. The calculator supports various computations on top 
of this structure, e.g., evaluating its value and printing its string
representation.

To evaluate the extensibility of a visitor based solution and an ITD based
solution we extend the calculator with two small extensions: first support
for subtraction and then support for an alternative string representation 
where constants are printed using natural language, i.e., one, two, 
three, etc.

%TODO: %Introduce the Expression problem.

Notice that this is a simplified example of an compiler. Information is
for instance only propagated bottom up so there is no need for parameter
propagation during a visit. Parameters would require a significantly more 
complex visitor due to the requirement on contravariance in the argument 
for type safety would require much more complex genericity. We chose an 
example that is flattering for visitors to be fair when arguing that ITDs 
are a more suitable solution even in this case. While the example is 
tiny the same structure is commonly used in compilers. In~\cite{aosd08abc} we argue 
the merits of using ITDs and attribute grammars rather than visitors in 
extensible compiler construction, using a complete Java compiler as a 
major case study.

\subsection{Visitor Implementation}
A visitor based solution uses the composite design pattern to model the
static structure of expressions: one class for each language element and
composition of language elements forming a tree structure. I.e., the 
Add node has references to its left and right child respectively, while 
the Int node has a single filed holding its value.

Each node type also contains an accept method that performs an additional
dispatch to select the code in the visitor for the concrete node type. 
This double dispatch scheme is the key to dispatching both on the node 
type and the visitor. We can thus select a particular method that is
selected based on both the dynamic type of the recieving object and the 
dynamic type of the visitor, provided as a parameter. 

A visitor is then simply a class that holds an implementation for each concrete 
class type that may accept the visitor. The Eval visitor computes the 
value for Int and Add, while the Print visitor computes a string 
represenation for each expression in a similar fashion. The visitor code 
is required to visit children indirectly using the accept method which 
clutters the implementation and is somewhat error prone.

\begin{lstlisting}[caption={Visitor Base}]
abstract class Expr {
}
class Int extends Expr {
  int value;
  <T> T accept(Visitor<T> v) {
    return v.visitInt(this);
  }
}
class Add extends Expr {
  Expr left;
  Expr right;
  <T> accept(Visitor<T> v) {
    return v.visitAdd(this);
  }
}
class Visitor<T> {
  <T> visitInt(Int i) { return null; }
  <T> visitAdd(Add a) { return null; }
}

class Eval extends Visitor<Integer> {
  Integer visitInt(Int i) {
    return i.value;
  }
  Integer visitAdd(Add a) {
    return a.left.accept(this) + a.right.accept(this);
  }
}
class Print extends Visitor<String> {
  String visitInt(Int i) {
    return Integer.toString(i.value);
  }
  String visitAdd(Add a) {
    return a.left.accept(this) + " + " + a.right.accept(this);
  }
}
\end{lstlisting}

\subsection{ITD Implementation}
The ITD implementation of the calculator has a similar implementation of the
node types as the visitor based solution, but the accept methods are no
longer necessary. Instead the aspect Eval introduces an eval method in each
node type. These methods can then be called from other introduced methods
directly without the need for using accept methods to enable dispatch.

\begin{lstlisting}[caption={ITD Base}]
abstract class Expr {
}
class Int extends Expr {
  int value;
}
class Add extends Expr {
  Expr left;
  Expr right;
}
aspect Eval {
  int Int.eval() {
    return value;
  }
  int Add.eval() {
    return left.eval() + right.eval();
  }
}
aspect Print {
  String Int.print() {
    return Integer.toString(value);
  }
  String Add.print() {
    return left.print() + " + " + right.print();
  }
}
\end{lstlisting}

\subsection{Extension}
As an example of extension we first add a subtraction node and enable the
existing analyses to support that new operation. Then, we add a new
analysis that computes a string representation of an expression where the
literals are printed using natural language.

\subsubsection{Visitor extension}
We need to add a new node type Sub that includes the boiler plate code to
support the visitor pattern. Then the base visitor is extended to support
the new node type. This change is not completely modular but a pragmatic
choice to keep the solution simple. We could have used various tricks with advanced
generics concepts, but that would make the solution even more complex and 
clutter the solution with framework code. It should be admitted that the 
current solution is not
completely safe since the default implementation for visitSub is not
suitable for the Eval and Print visitors. 

We also provide extended vistitors
,ExtEval and ExtPrint, to support the extended language.
They can extend the base behaviour and simply provide the increment
needed to support the subtraction operation. The extended version of the
printer can then be refined to display integer constants using natural
language in the separate visitor WordPrint. Notice that functionality from
the base visitor can be reused here as well and only the functionality 
that is refined need to be provided.

\begin{lstlisting}[caption={Visitor Extension}]
class Sub extends Expr {
  Expr left;
  Expr right;
  T accept(Visitor<T> v) {
    return v.visitSub(this);
  }
}
class Visitor<T> { // replace old impl.
  T visitInt(Int i) { return null; }
  T visitAdd(Add a) { return null; }
  T visitSub(Sub s) { return null; } // new
}
class ExtEval extends Eval {
  Integer visitSub(Sub s) {
    return a.left.accept(this) - a.right.accept(this);
  }
}
class ExtPrint extends Print {
  String visitSub(Sub s) {
    return a.left.accept(this) + " - " + a.right.accept(this);
  }
}
class WordPrint extends ExtPrint {
  Integer visitInt(Int i) {
    .. convert i to the strings "one", "two", etc.
  }
}
\end{lstlisting}

\subsubsection{ITD extension}
The ITD-based solution simply adds a new node type and introduces methods in that
node type to make the evaluator and printer support subtraction.
The natural language based printer is slightly more complicated. The simple
model of introducing new methods in an existing class is not as convenient
in this case. We can not keep the implementation printing numbers and reuse
the other printing functionality while still changing it to print using
natural language. The word printer therefore choses to take precedence over
the other definitions of \texttt{print} making it the only implementation, thus
disabling the normal printer.

\begin{lstlisting}[caption={ITD Extension}]
class Sub extends Expr {
  Expr left;
  Expr right;
}
aspect ExtEval {
  int Sub.eval() {
    return left.eval() - right.eval();
  }
}
aspect ExtPrint {
  String Sub.print() {
    return left.print() + " - " + right.print(); 
  }
}
aspect WordPrint {
  declare precendence: WordPrint, *;
  String Int.print() {
    .. convert i to the strings "one", "two", etc.
  }
}
\end{lstlisting}

\subsection{Discussion}
While both presented solutions are quite similar there are some important
differences:

Visitors require some ahead of time planning with boiler plate code in the 
node types while the ITD solution can modularly update an existing class 
hierarchy if needed. The actual visitors can be updated in a 
modular fashion using inheritance and overriding but the user of these 
analyses need to update the code to use the extended versions. ITDs can 
be used to directly add support for existing analyses in the new node type 
in a completely modular fashion. 

The visitor based solution need to include framework code to enable the 
double dispatch pattern which clutters the implementation and is somewhat 
error prone. The ITD solution on the other hand is quite 
straightforward with a clean implementation.

The base visitor could not be modularly extended in the presented example.
There are solutions based on advanced usage of generics that support
modular extension but that requires much more framework code and clutters
the solution even more. Moreover, the user need still replace uses of the
old visitors with uses of the new visitor.
(as long as we don't try to run Eval
or Print on a tree containing sub nodes. could probably be fixed using more
complex visitor)

Visitors can be inherited and overridden to refine behaviour in a visitor
being extended. This enables both visitors to be used in the same system
while the ITD solution can only use either the original implementation 
or the refined implementation. 

%%Removed for now, we don't address this issue in the module solution.
%Another advantage of the visitor based solution is that each class can be
%modularly type-checked and compiled while the ITD based solution requires a
%global analysis since methods can be introduced by any aspect in the
%system.


\section{Module System for ITDs}
\label{section:jastaddmodules}
This provides a description of the JastAdd module system.

\subsection{Module System Overview}

A module consists of aspects and classes, and defines the set of external
modules that are visible to the module's members. The module system is based
on the object-oriented java module system presented in \cite{modulesastypes}

\subsection{Declaration, Membership and Exports}

A module is defined in a \texttt{.module} file, and is headed by the module's name:

\begin{lstlisting}[caption={Module Declaration}]
//file prettyprinter.module
module prettyprinter;
...
\end{lstlisting}

Membership to a module is defined in the compilation unit of the members,
similar to a package declaration:

\begin{lstlisting}[caption={Module Membership}]
//file PrettyPrinter.jrag
module prettyprinter; //module membership
public aspect PrettyPrinter {
	public abstract String Expr.prettyPrint();
	
	public String Add.prettyPrint() {
		return getLeft().prettyPrint() + 
				"+" + getRight().prettyPrint();
	}
	
	public String IntLit.prettyPrint() {
		return getIntLit().toString();
	}
}

//file Expr.java
module asttypes; //module membership
package expr; //package declaration
public class Expr {
...
}
\end{lstlisting}

Module and package declarations can coexist in a single compilation unit.
Modules can contain aspects and classes that span multiple packages.

Packages are also not implicitly visible outside the module unless an
\texttt{export} declaration is provided for that package. These declarations
are placed in \texttt{.module} files.

\begin{lstlisting}[caption={Export Package}]
//file asttypes.module
module asttypes;
export package expr, stmt; //export expr and stmt
...

//file prettyprinter.module
module prettyprinter;
export package *; //export all packages
...
\end{lstlisting}

As the example shows, an export package may contain a list of packages, or the
wildcard \texttt{*}, which exposes all packages in the module. Any types that
belong to a package that is not exported are not visible from outside the module.

\subsection{Imports and Instantiation}

Module definitions also contain import declarations, which specify which
other modules are visible to the members of the module. A module imports 
an \textit{instance} of another module, which allows multiple instances
of the same module to exist within a single context.

There are two ways to import a module instance: importing the singleton or
an \textbf{own} instance. Imports may also be alised to allow multiple instances
of the same module to exist in the same context, and be exported so that
they are also accessible through indirection from other modules;

\begin{lstlisting}[caption={Imports}, label={figure:imports}]
//file prettyprinter.module
module prettyprinter;
//import the singleton instance of module parserframework
import parserframework; 
//import an own instance of the module asttypes
import own asttype;
//import another instance of asttype, using an alias and export
import own asttype export as public_ast;
\end{lstlisting}

Imports of \textbf{own} instances without an explicit alias use the name
of the imported module as its alias. To be more specific, line 6 from the above
example is equivalent to
\begin{lstlisting}
import own asttype as asttype;
\end{lstlisting}

The visibility of imported modules become important for the \textbf{merge}
operation, described later.

\subsection{Lookup}

Allowing two instances of the same module to exist in the same context
requires a way to disambiguate references to the members of the instances.
We introduce \textit{module qualifiers} for type references to explicitly
choose the module to which the type belongs.

The following example uses the module definition from listing \ref{figure:imports}.
The module \texttt{prettyprinter} imports two instances of \texttt{asttype}, with the
aliases \texttt{asttype} and \texttt{public\_ast}. The aspect \texttt{PrettyPrinter},
which belongs to the module, introduces ITDs to each of the \texttt{Add} types
of the instances.

\begin{lstlisting}[caption={Module Qualfiers}]
//file PrettyPrinter.jrag
module prettyprinter;
aspect PrettyPrinter {
	...
	//modifies the Add type in asttypes
	public String asttypes::Add.prettyPrint() {
		//infix print
		return getLeft().prettyPrint() + "+" + 
				getRight().prettyPrint();
	}
	...
	//modifies the Add type in public_ast
	public String public_ast::Add.prettyPrint() {
		//postfix print
		return getLeft().prettyPrint() + 
				getRight().prettyPrint() + "+";
	}
}
\end{lstlisting}

\subsection{Merge}



\subsection{Extension}



\subsection{ITD Calculator with Modules}

\subsection{Evaluation}



\section{Case Study}
\label{section:casestudy}
This describes the case study on the Java1.4 frontend and
backend, and how the modules fit in.

\section{Implementation}
\label{section:implementation}
This describes the implementation.

Modular extension to JastAdd compiler based on JastAddJ.

concrete and abstract syntax for module 

Name mangling.

name lookup
  filter
  mangling

the same functionality that makes name binding extensible according to our
previous papers also makes it easier to perform the modified name lookup
for modules.

compile-time solution. 

generate new packages for different instances.

compile-time for JastAddJ?







\section{Related Work}
\label{section:related}
This is the related lit section.

iJAM for module instantiation \cite{iJAM}.

Modules as OO Types \cite{modulesastypes}.

JastAdd \cite{jastadd}

Open Modules for AJ \cite{openmodulesaj}

CaesarJ for aspect deployment \cite{caesarj}

Aspectual collaborations \cite{lieberherr03aspectual}

Aspectual mixin layers (for aspect refinement) \cite{aspectualmixinlayers}.
Ditto for virtual classes \cite{virtualclasses89}

Hyper/J for merge \cite{hyperj}

AOP and modularity \cite{aopmodularreasoning}

FOP \cite{fopstepwiserefinement}

XPIs (in case interfaces come up) \cite{xpi}

AspectJ \cite{overviewaspectj}

Composition filters \cite{compositionfilters}

\section{Conclusions}
\label{section:conclusions}
We have presented a module system that improves extensibility and
information hiding when using ITDs. Instantiation enables multiple
variants of a base system, each with its own set of applied ITDs. Imports
and exports enable better support for access control improving information
hiding. An important consequence of these features is that they remove some
deficiencies of using plain ITDs compared to a visitor based solution.

The module system has been implemented as an extension to the JastAdd
system, an aspect-oriented attribute grammar extension to Java, which
supports ITDs as one of its major modularization mechanisms. The solution
is purely compile-time and generates class files that can be executed on
a traditional JVM without explicit support for modules.

To evaluate the module system on a realistic system we retrofitted the
JastAdd Extensible Java Compiler (JastAddJ) to use modules. It is a medium
sized application of more than 21.000 lines of code which is designed using
the somewhat extreme viewpoint that the class hierarchy does not contain
any behavior but modularly added using ITDs. Minor changes to the
implementation enabled us to create a family of compilers that can coexist
in the same system using modules compared to the previous solution using an
ugly build file hack. Moreover, it enabled us to limit the scope of many
properties that were global in the original implementation.

As future work we will experiment with using the module instantiation
features to further enhance the reuse of ITDs. One can for instance
envision a situation where ITDs are multiply instantiated in different
modules of the same compiler rather than only across tool boundaries.


%ACKNOWLEDGMENTS are optional
%\section{Acknowledgements}
%\label{section:acknowledgements}
%\input{text/acknowledgements}

%
% The following two commands are all you need in the
% initial runs of your .tex file to
% produce the bibliography for the citations in your paper.
\bibliographystyle{abbrv}
\bibliography{sigproc}  % sigproc.bib is the name of the Bibliography in this case
% You must have a proper ".bib" file
%  and remember to run:
% latex bibtex latex latex
% to resolve all references
%
% ACM needs 'a single self-contained file'!
%
\end{document}
