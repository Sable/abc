As a further demonstration of the proposed module system's ability to 
express module specifications that are present in existing systems, 
we provide a simple translation from OSGi bundles to the proposed modules.
The translation is then run on a JHotdraw project refactored to use
OSGi bundles.

\subsection{OSGi Bundles}

First we provide a brief description of OSGi bundle system, and its basic
features that relate to bundle composition and the constraints thereon.
OSGi bundle constraints are stored in a JAR manifest file, and has the
general format

\begin{lstlisting}
attribute : value [;constraint]
\end{lstlisting}

A bundle has a \textit{symbolic name} which acts as the unique identifier for that bundle, 
and can also contain a more descriptive name.
It also has a version number which is used to distinguish differing versions of
a bundle with the same symbolic name, as well as a list of exported packages.
 The bundle infomation stored in the
manifest file applies to all classes that reside in the JAR. The following example
bundle information has the symbolic name \textit{org.apache.batik}, version \textit{1.6.0},
and exports a set of packages.

%basic bundle (symbolic-name, version, singleton}
\begin{lstlisting}[caption=Basic OSGi bundle]
Bundle-Name: Batik fragment for JHotdraw taken from v1.6
Bundle-SymbolicName: org.apache.batik
Bundle-Version: 1.6.0
Export-Package: org.apache.batik, org.apache.batik.ext.awt,
 org.apache.batik.ext.awt.image, org.apache.batik.ext.awt.image.renderable,
 org.apache.batik.ext.awt.image.rendered, org.apache.batik.util
\end{lstlisting}

%require-bundle
%import-package
Module composition can be done in two ways: first through a \textit{Require-Bundle}
entry which explicitly contains the symbolic name of the bundle dependency, or an
\textit{Import-Package} entry which only contains a set of package names, and leaves
it up the to framework which bundle to use to satisfy the dependency. The \textit{Require-Bundle}
entry may also have a version constraint, which is a range of versions that can satisfy
the dependency.

\begin{lstlisting}[caption=Require-bundle and Import-package]
Bundle-Name: JHotdraw GUI Application Framework
Bundle-SymbolicName: org.jhotdraw
Bundle-Version: 1.0.0
Require-Bundle: net.n3.nanoxml;bundle-version="[1.0.0,2.0.0)"
Import-Package: org.apache.batik, org.apache.batik.ext.awt,
 org.apache.batik.ext.awt.image, org.apache.batik.ext.awt.image.renderable,
 org.apache.batik.ext.awt.image.rendered
\end{lstlisting}

%singleton
Bundles may also be declared \textit{singleton}, which means that for all
singleton bundles with the same symblic name, only one instance is allowed to
exist. This, however, does not affect non-singleton bundles, even if they have
the same symbolic name.

\begin{lstlisting}[caption=Singleton Bundle]
Bundle-SymbolicName: org.apache.batik;singleton:=true
\end{lstlisting}

%wiring
Upon instantiation, the OSGi framework \textit{wires} the bundle to its dependencies.
It connects the bundle instance to a specific instance of its bundle dependency, making
sure that any constraints are satisfied. If it is unable to find a bundle to wire to a
dependency, the instantiation of the bundle fails.

\subsection{Translation}

%bundle to module mapping

%wiring
