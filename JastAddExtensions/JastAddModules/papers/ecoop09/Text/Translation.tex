As a further demonstration of the proposed module system's ability to 
express module specifications that are present in existing systems, 
we provide a simple translation from OSGi bundles to the proposed modules.

\subsection{OSGi Bundles}

First we provide a brief description of OSGi bundle system, and its basic
features that relate to bundle composition and the constraints thereon.
OSGi bundle entries are stored in a JAR manifest file, and has the
general format

\begin{lstlisting}
attribute : value [;directive]
\end{lstlisting}

A bundle has a \textit{symbolic name} which acts as the unique identifier for that bundle, 
and can also contain a more descriptive name.
It also has a version number which is used to distinguish differing versions of
a bundle with the same symbolic name, as well as a list of exported packages.
 The bundle infomation stored in the
manifest file applies to all classes that reside in the JAR. The following example
bundle information has the symbolic name \textit{org.apache.batik}, version \textit{1.6.0},
and exports a set of packages.

%basic bundle (symbolic-name, version, singleton}
\begin{lstlisting}[caption=Basic OSGi bundle]
Bundle-Name: Batik fragment for JHotdraw taken from v1.6
Bundle-SymbolicName: org.apache.batik
Bundle-Version: 1.6.0
Export-Package: org.apache.batik, org.apache.batik.ext.awt,
 org.apache.batik.ext.awt.image, org.apache.batik.ext.awt.image.renderable,
 org.apache.batik.ext.awt.image.rendered, org.apache.batik.util
\end{lstlisting}

%require-bundle
%import-package
Module composition can be done in two ways: first through a \textit{Require-Bundle}
entry which explicitly contains the symbolic name of the bundle dependency, or an
\textit{Import-Package} entry which only contains a set of package names, and leaves
it up the to framework which bundle to use to satisfy the dependency. The \textit{Require-Bundle}
entry may also have a version constraint, which is a range of versions that can satisfy
the dependency.

\begin{lstlisting}[caption=Require-bundle and Import-package]
Bundle-Name: JHotdraw GUI Application Framework
Bundle-SymbolicName: org.jhotdraw
Bundle-Version: 1.0.0
Require-Bundle: net.n3.nanoxml;bundle-version="[1.0.0,2.0.0)"
Import-Package: org.apache.batik, org.apache.batik.ext.awt,
 org.apache.batik.ext.awt.image, org.apache.batik.ext.awt.image.renderable,
 org.apache.batik.ext.awt.image.rendered
\end{lstlisting}

%singleton
Bundles may also be declared \textit{singleton}, which means that for all
singleton bundles with the same symblic name, only one instance is allowed to
exist. This, however, does not affect non-singleton bundles, even if they have
the same symbolic name.

\begin{lstlisting}[caption=Singleton Bundle]
Bundle-SymbolicName: org.apache.batik;singleton:=true
\end{lstlisting}

%wiring
Upon instantiation, the OSGi framework \textit{wires} a bundle to its dependencies.
Wiring involves connecting the bundle instance to a specific instance of its bundle dependency, making
sure that any constraints are satisfied. If it is unable to find a bundle to wire to a
dependency, the instantiation of the bundle fails.

\subsection{Translation}

Translation from OSGi bundles to modules in the proposed system has two general phases:
first the generation of modules and module interfaces that correspond to the bundles and
their \textit{Require-Bundle} and \textit{Import-Package} entries, and then the generation of
a \textit{system} module, which wires each module to its dependencies.

\subsubsection{Bundle to Module Mapping}

%bundle to module mapping
Each bundle is translated to a module declaration with a name that contains the bundle's
symbolic name and version. This turns every bundle version into a first class entity which
can be referenced by using the module-level language specified in section \ref{moduletypes}.
The generated module will contain the same set of exported packages as the bundle.

The next step is to generate a non-weak module interface for each unique \textit{symbolicName, versionConstraint}
pair in the set of \textit{Require-Bundle} entries for the entire compilation. These will be inserted 
as import module declarations in the translated modules, and all modules that satisfy the \textit{versionConstraint}
will be made to implement the generated interface.

Translating the \textit{Import-Package} entries first involve partitioning the set of imported 
packages into sets that belong to a single module, and then converting each these partitions into
weak module interfaces. This partitioning is done by a heuristic: packages that have the same first
three segments belong to the same partition. If a package name has less than three segments, then
it belongs to its own partition. As an example, the set of packages\\
\[
\{net.n3.nanoxml, org.jhotdraw, org.jhotdraw.app, org.jhotdraw.app.action\}
\]
is partitioned into\\
\[\{net.n3.nanoxml\}, \{org.jhotdraw\}, \\ \{org.jhotdraw.app, org.jhotdraw.app.action\}\]

This partition is necessary to ensure that the packages are wired to the same module. OSGi provides
the \textit{uses} directive to specify this, but we exclude processing this directive during translation 
for the sake of simplicity.

The following example demonstrates the module generation procedure. The example contains three bundles
\textit{org.jhotdraw}, \textit{net.n3.nanoxml} and \textit{org.apache.batik}. The bundle \textit{org.jhotdraw}
depends on the other two modules, explicitly on \textit{org.apache.batik} through a \textit{Require-Bundle}, 
and implicitly on \textit{net.n3.nanoxml} through an \textit{Import-Package} entry.

\begin{lstlisting}[caption=Bundle Translation Example]
//bundle org.jhotdraw
Bundle-SymbolicName: org.jhotdraw
Bundle-Version: 1.0.0
Import-Package: net.n3.nanoxml
Require-Bundle: org.apache.batik;bundle-version="[1.6.0,1.8.0)"
Export-Package: org.jhotdraw, org.jhotdraw.app, ...
...
//bundle org.apache.batik
Bundle-SymbolicName: org.apache.batik;singleton:=true
Bundle-Version: 1.8.0.pre
Export-Package: org.apache.batik, org.apache.batik.ext.awt,...
...
//bundle net.n3.nanoxml
Bundle-SymbolicName: net.n3.nanoxml
Bundle-Version: 1.0.0
Export-Package: net.n3.nanoxml
\end{lstlisting}

Following the algorithm outlined above, the bundles are translated to the following set of modules:

\begin{lstlisting}[caption=Translated Bundles]
//file org.jhotdraw_1_0_0.module
//translated from bundle org.jhotdraw, version 1.0.0
module org.jhotdraw_1_0_0;
import own I_org.apache.batik_v_1_6_0__1_8_0 export as I_org.apache.batik_v_1_6_0__1_8_0;
import own WI_importpackage_org.jhotdraw_0 export as WI_importpackage_org.jhotdraw_0;
export package org.jhotdraw, org.jhotdraw.app, ...;

//file org.apache.batik_1_8_0_pre.module
//translated from org.apache.batik, version 1.8.0.pre
module org.apache.batik_1_8_0_pre	implements I_org.apache.batik_v_1_6_0__1_8_0;
export package org.apache.batik, org.apache.batik.ext.awt,...;

//file net.n3.nanoxml_1_0_0.module
//translated from bundle net.n3.nanoxml, version 1.0.0
module net.n3.nanoxml_1_0_0;
export package net.n3.nanoxml;

//file I_org.apache.batik_v_1_6_0__1_8_0.module
//generated from a Require-Bundle entry
module_interface I_org.apache.batik_v_1_6_0__1_8_0;
export package org.apache.batik, org.apache.batik.ext.awt, ...;

//file WI_importpackage_org.jhotdraw_0.module
//generated from Import-Packge entry
weak_module_interface WI_importpackage_org.jhotdraw_0;
export package net.n3.nanoxml;
\end{lstlisting}

\subsubsection{System Module Generation}

%wiring
To simulate the wiring done by the OSGi runtime, a top-level module \textit{system} that imports all 
the concrete modules is generated and populated with the apppropriate \textit{replace} declarations to connect modules to
their dependencies. The generation of the \textit{system} module also takes \textit{singleton}
bundles into account.

Continuing the example above, the following \textit{system} module is generated by the translation:

\begin{lstlisting}[caption=Generated System Module]
module system;
import own org.jhotdraw_1_0_0 export as org.jhotdraw_1_0_0;
import own net.n3.nanoxml_1_0_0 export as net.n3.nanoxml_1_0_0;
import own org.apache.batik_1_8_0_pre export as org.apache.batik_1_8_0_pre;

replace org.jhotdraw_1_0_0::I_org.apache.batik_v_1_6_0__1_8_0
	with org.apache.batik_1_8_0_pre;
replace replace org.jhotdraw_1_0_0::WI_importpackage_org.jhotdraw_0
	with net.n3.nanoxml_1_0_0;
\end{lstlisting}

The translation demonstrates that the proposed module system can express the basic import
constraints in OSGi bundles. There are also additional matching criteria such as
export package attributes that are supported by OSGi, but it is also possible to simulate
these by generating the appropriate set of module interfaces.

The translation itself is used on a case study on JHotdraw modified to use OSGi bundles.
Both the implementation of the translation and the translation case study are available at
\textit{http://progtools.comlab.ox.ac.uk/members/neil}.