

%Versioning is done through constraints, usually flat

%Extension is specified without reference to versioning

%Limited and cumbersome support for constraints at the publisher side

%

The paper is organized as follows. First we describe the
usage and shortcomings of existing module systems in section \ref{moduleops}. 
We then define a module system that contains relations and operations 
that closely resemble object-oriented types in section \ref{moduletypes}. 
In section \ref{eval} we demonstrate that the module
system is able to express existing usage of modules while providing new
capabilities to handle split packages, backwards compatibility and module extension. The module system
itself is implemented in a compiler made using the JastAdd\cite{jastadd} compiler construction
framework, and is demonstrated on a small case study on JHotdraw 7.1 \cite{jhotdraw}.
To demonstrate that the module system contains the functionality provided by existing
module systems, we provide a translation from OSGi bundles, a module system in widespread
use for Java, into the proposed module system in section \ref{translation}.
A short description of the implementation is included in section \ref{implementation}.
The paper then ends with sections on related work and conclusions.
