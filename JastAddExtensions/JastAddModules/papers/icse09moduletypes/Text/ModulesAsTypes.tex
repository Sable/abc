%------------------------------------------------------------------------- 

As the previous section shows, while the problem of Java modules have been
well studied and multiple solutions exist for different aspects of the problem,
there are still remaining issues such as split packages, versioning constraints 
and extension that remain. We propose that these can be solved by
defining a type system on modules, which will model the current usage
of modules while allowing for the expression of constraints and relations
that were previously not possible.

We now non-formally define a type system over collections of classes. 
A module is a collection of classes, and we consider it to be the type of 
the collection of classes that belong to it. We then proceed to define
relations and operations on modules and, in a later section, show how
these model current module usage. 

Most of the examples are abridged versions taken from the JHotdraw case study, 
which is available at \textit{http://progtools.comlab.ox.ac.uk/members/neil}. 

\SubSection{Declaration and Membership}

Modules are defined in a {\tt .module} file that begins with a module
declaration:

\begin{lstlisting}
//File org.jhotdraw.module
module org.jhotdraw; ...
\end{lstlisting}

Module membership is declared using syntax that was proposed 
for JSR 294 \cite{superpackageChange}. Membership of a compilation unit is 
declared by adding a module declaration. If a package declaration is present, 
the package name is used as the package name of the file within the module.

\begin{lstlisting}
//File Gradient.java
module org.jhotdraw;
package org.jhotdraw.samples.svg;
public interface Gradient extends Cloneable { ... }
\end{lstlisting}

Split packages cause hard to find problems in current module systems\cite{iJAMComments}. To avoid this, 
a way to specify package membership in a module allowed by declaring the module membership
in the {\tt package-info.java} file of that package.

\begin{lstlisting}
//File org/jhotdraw/app/action/package-info.java
//all classes in package org.jhotdraw.app.action also
//belong to module org.jhotdraw
module org.jhotdraw;
package org.jhotdraw.app.action;
\end{lstlisting}

If the module membership specified in a compilation unit's {\tt package-info.java}
file is in conflict with that specified in the file itself, the membership
declared in the file takes precedence, but a warning is issued to flag a possibly
split package.

%------------------------------------------------------------------------- 
\SubSection{Imports}

For a module to be able to access any of another module's classes,
it must first import an instance of that module. This way, imports form
a hard constraint on the visibility of classes.

There are two ways to import a module. The first is to import the singleton
instance of the module. This is done in the following manner:

\begin{lstlisting}
//imports the singleton instance
import org.jhotdraw;
\end{lstlisting}

Importing the singleton instance of a module allows you to share the classes
of that module with other modules that also import it.

Similar to iJAM \cite{iJAM}, a module may also import its \textbf{own} instance of
a module. It is also allowed to rename that instance to allow for multiple
instances of the same module to exist in the same context.

\begin{lstlisting}
//imports an own instance
import own org.jhotdraw;
//imports on own instance with an alias
import own org.jhotdraw as jhotdraw;
\end{lstlisting}

Unlike iJAM, renaming an instance does not implicitly allow other
modules to access that instance. To allow access to other modules, the import
must be exported:

\begin{lstlisting}
//imports an own instance, and makes it available to
//other modules
import own org.jhotdraw export as jhotdraw;
\end{lstlisting}

The visibility of the module instances become important for module
qualified name lookups and the merge and replace operations described in later
sections.

An import declaration creates a typed module reference that points to the module
instance specified. The type of the reference created is the module
in the import declaration. The module reference can later be pointed to another
instance using the operations \textbf{replace} and \textbf{merge} described
in the following sections.

Import cycles involving \textbf{own} instances are not allowed, as this would 
cause an infinite loop of module instantiation. Import cycles involving only
singleton modules, however, are allowed.

\SubSection{Exported packages}

We allow exported packages similar to {\tt export-package} in OSGi bundles. Unlike OSGi, however,
exporting a package does not automatically allow other modules to gain
access to these packages. A module A must first import another module B
before gaining access to B's exported packages.

Exported packages are declared in the {\tt .module} file:

\begin{lstlisting}
module org.jhotdraw.batikfragment;
//allow other modules access to types in these packages
export package 
	org.apache.batik.ext.awt.image,
	org.apache.batik.ext.awt,
	org.apache.batik;
\end{lstlisting}

You may also export all the packages that belong to a module by using
the * wildcard. This is the only way to export classes that belong to
the default package in a module.

\begin{lstlisting}
module org.jhotdraw;
...
//export all packages
export package *;
\end{lstlisting}

%We also allow the \textbf{module} access modifier proposed for JSR294 \cite{JSR294}
%which applies to classes, fields and methods. Module private access only allows
%accesses from classes that belong to the same module.

%\begin{lstlisting}
%module org.x.y; //module membership
%module class C { //only visible in org.x.y
%	public void publicMethod() {...};
%	module void moduleMethod() {...}; //same here
%}
%\end{lstlisting}

\SubSection{Module Qualified Type References}

Type references can now be qualified with module names to explicitly
identify the module that contains the class. Module qualifiers
are separated from package and type names by the \texttt{::} separator,
and also allows for indirection. For example, given the following module:

\begin{lstlisting}
module org.jhotdraw;
import own net.n3.nanoxml export as nanoxml;
import own org.apache.batik export as batik;
export package *;
\end{lstlisting}

The class \texttt{org.jhotdraw.Version} in the module \texttt{org.jhotdraw}
can be accessed as
\begin{lstlisting}
org.jhotdraw::org.jhotdraw.Version
\end{lstlisting}
Classes in \texttt{nanoxml} and \texttt{batik} can be accessed by indirection:
\begin{lstlisting}
org.jhotdraw::nanoxml::net.n3.nanoxml.Version
org.jhotdraw::batik::org.apache.batik.Version
\end{lstlisting}

In the following example, a sample application \texttt{org.jhotdraw.samples.simple}
uses the Jhotdraw framework by importing its own instance of the module
\texttt{org.jhotdraw}. The module \texttt{org.jhotdraw} itself uses the 
two libraries \texttt{net.n3.nanoxml} and \texttt{org.apache.batik}, and
exports them so that other modules can also access their classes. The 
application then gets the versions of JHotdraw and its dependencies.

\begin{lstlisting}[caption=Module Qualified Type References,numbers=left]
//File org.jhotdraw.module
module org.jhotdraw;
import own net.n3.nanoxml export as nanoxml;
import own org.apache.batik export as batik;
export package *;

//File org.apache.batik.module
module org.apache.batik;
export package 
	org.apache.batik.ext.awt.image,
	org.apache.batik.ext.awt,
	org.apache.batik;

//File net.n3.nanoxml
module net.n3.nanoxml;
export package net.n3.nanoxml;

//File org.jhotdraw.samples.simple.module
module org.jhotdraw.samples.simple;
//import own instance of jhotdraw
import own org.jhotdraw export as jhotdraw;
export package *;

//File Main.java
module org.jhotdraw.samples.simple;
package org.jhotdraw.samples.simple;
//on-demand import of indirect import batik
import jhotdraw::batik::org.apache.batik.*;
public class Main {
	//direct imports don't need a qualifier
	String jhotdrawVersion = 
		org.jhotdraw.Version.version;
	//module qualified type name
	String nanoXMLVersion = 
		jhotdraw::nanoxml::net.n3.nanoxml.Version.version;
	//from the on-demand import
	String batikVersion = Version.version;
}
\end{lstlisting}

The class \texttt{org.jhotdraw.Version} can be accessed without
a module qualifier since it is a member of the JHotdraw module, 
which is directly imported by the application Simple. To access
the versions of JHotdraw's dependencies, the type names have to
be qualified with the module qualifier that denotes their parent
module. As \texttt{net.n3.nanoxml.Version} is not a direct import,
it needs to be qualified with \texttt{jhotdraw::nanoxml::}, which
points to the \texttt{nanoxml} import of JHotdraw. The same is
true for \texttt{org.apache.batik.Version}, and as the example shows,
the module qualified names can also be used in import declarations.

Module qualifiers allow a developer to explicitly define from which module
to load the class. This enables to disambiguation of ambiguous type
lookups that are caused by split packages, or by a module importing
two different versions of the same module.

The module qualifiers are relative and are looked up in the context 
of the module that contains them. This removes the need for globally
unique names which would have been difficult if not impossible in the
presence of \textbf{own} module instances. Also, a module can only
access a module reference of its imported module if the imported module
exported that reference. This allows a module to hide dependencies from
its clients. This becomes useful when module interfaces are introduced
in a later section.

Types that do not belong to a module are considered to be members of
the ``default'' module. This module has a blank name, and its singleton
instance is implicitly imported by all modules. So, a module-less
type {\tt javax.swing.Action} can be accessed in this way:

\begin{lstlisting}[caption=Default Module Lookups]
module somemodule;
package mypackage;
//import from the default module
import ::javax.swing.Action;
class MyClass {
	Action a = new Action(){...};
}
\end{lstlisting}

It is not strictly necessary to always module qualify accesses to types in
the default module, as long as there are no classes in the current module
or its imported modules that have the same package and type name. However,
if these do exist, it is necessary to use the module qualified name to access
these classes to avoid an ambiguous type error.

%------------------------------------------------------------------------ 
\SubSection{Subtyping}

We now define a subtyping relation on the module type. A module is declared
to be the subtype of another module by using the \textbf{extends} keyword:

\begin{lstlisting}
module defaultsample.defaultjhotdraw 
	extends org.jhotdraw;
\end{lstlisting}

A subtype module inherits all the import declarations and export package declarations 
of its parent module, and can add some more of its own. It also inherits the member 
compilation units (and the types therein) of its supertype.

If a subtype module contains a type with the identical package and type name
to another type which is a member of its supertype module, 
this shadows the type in the supertype module for all type references that
originate from the subtype module, and any other modules that import an
instance of the subtype module.

The following example demonstrates module subtyping and shadowing. The module
\texttt{defaultsample.defaultjhotdraw} extends the module \texttt{org.jhotdraw}. 
It also contains the class \texttt{org.jhotdraw.app.action.AboutAction},
which overrides a class with the same package and name in the supertype module.
The new \texttt{AboutAction} extends the version in the supertype module and
changes the message of the about dialog.

\begin{lstlisting}[caption=Module Subtyping]
//file defaultsample.defaultjhotdraw.module 
module defaultsample.defaultjhotdraw 
	extends org.jhotdraw;
//the batik and nanoxml imports are 
//inherited from org.jhotdraw
export package *;

//file org/jhotdraw/app/action/AboutAction.java
module org.jhotdraw;
package org.jhotdraw.app.action;
//this class will be overridden
public class AboutAction 
		extends AbstractApplicationAction {
	...
	public void actionPerformed(ActionEvent evt) {
		...
		JOptionPane.showMessageDialog(...);
	}
}

//file defaultjhotdraw/app/action/AboutAction.java
module defaultsample.defaultjhotdraw;
package org.jhotdraw.app.action;
import batik::org.apache.batik.Version;
//This class shadows the class
//org.jhotdraw::org.jhotdraw.app.action.AboutAction
public class AboutAction 
	extends 
	supermodule::org.jhotdraw.app.action.AboutAction {
	public void actionPerformed(ActionEvent evt) {
		...
		//add batik version to about message
		JOptionPane.showMessageDialog(...
			+ "Batik version " + Version.version,
			...);
	}
}

\end{lstlisting}

As the example shows, the overriding \texttt{AboutAction} class extends the
overridden class in the supertype module by using the special 
\texttt{supermodule::} module qualifier to access the overridden class. It
then redefines the only method required to change the behavior as required.
All lookups for the class \texttt{AboutAction} now resolve to the overriding
class, including those in the classes that belong to the supertype module.

Without module subtyping, changing \texttt{AboutAction} without access to the source
would require extending the class and reimplementing the method 
that instantiates it. Module extension allows the same extension without
any reimplementation except for the code that actually changes the behavior of the overridden class.
In cases such as that above, extension using module subtyping could result in savings in
terms of reimplemented framework code by moving extension and
composition to the higher level of abstraction provided by the module.

%------------------------------------------------------------------------- 
\SubSection{Overrides}

Subtyping allows extension or patching of a module without actually
rebuilding a completely new module. However, it does have the disadvantage
of being dependent on the existence of its supertype modules. To get
around this limitation, we define the \textbf{overrides} relation to allow
a module to completely replace another module. A module is declared
to override another using the \textbf{overrides} keyword:

\begin{lstlisting}
module org.apache.batik1_8pre 
	overrides org.apache.batik1_6, org.apache.batik1_7;
\end{lstlisting}

Unlike subtyping, an overriding module does not inherit anything from the
modules it overrides. However, it \textit{must} provide the same exported modules
and packages as its overridden modules to satisfy the external and internal
clients of these modules.

As is expected, override is inherited by subtype modules. This allows a subtype
of an overriding module to override the same modules as its supertype.

Declaring an overriding module does not automatically change over all references
to the overridden module to use the overriding module. This is done using the
\textbf{replace} operation described in the section below, which also contains
an example of the use of \textbf{overrides}.

%------------------------------------------------------------------------- 
\SubSection{Replace}

One of the common operations on modules is replacing an older version of a
dependency with a newer one. In current module systems, this is a global operation,
the new versions are added to the repository and the clients then use the latest
compatible version available. Since the proposed module type system allows for instances
of a module that are local to its client, we define an operation \textbf{replace} which
changes the target of a list of module references in the context of a module:

\begin{lstlisting}
replace m1, m2::m3, m4, ... with <moduleexpr>;
\end{lstlisting}

The replace operation changes a list of module references to point to an instance
defined by a module expression. A module expression may be a module reference to
an existing instance, a reference to the singleton instance of a module type, or
a new \textbf{own} instance of a module type.

\begin{lstlisting}[caption=Module Expressions for Replace]
//replace with an existing reference
import own org.apache.batik1_6 as batik;
replace jhotdraw::batik with batik;
//replace with a the singleton instance
replace jhotdraw::batik 
	with singleton org.apache.batik1_6;
//replace with a new instance of a module
replace jhotdraw::batik with own org.apache.batik1_6;
\end{lstlisting}

A module reference can be replaced with a module expression if its static type
is the same, is a supertype or is overridden by the type of the module expression.

Replace declarations are also inherited by subtype modules. The replace sequence for a
module is given by a list of its supertype's replaces, starting from the farthest
ancestor. There is no way to exclude a supertype's replaces, as
this may reduce the module signature (the set of module references available) of
the supertype module, which could cause the internal and external clients of
the supertype to break.

It should be noted that an instance created using the \textbf{own} module expression is not
implicitly imported by the module to which the replace declaration belongs. This
is useful when updating the dependencies of an import when the module itself
does not require the same dependency.

The following example taken from the JHotdraw case study shows how replace can
be used to update a dependency to a more recent version. The example contains
the sample JHotdraw applications \texttt{samples.net} and \texttt{samples.netold}, 
where \texttt{samples.net} is a newer version of the other application. They both 
derive from \texttt{samples.defaultsample}, which imports \texttt{defaultsample.defaultjhotdraw} 
from the extends example in the previous section. The sample \texttt{netold} updates 
the \texttt{batik} dependency of \texttt{jhotdraw} to the 1.7 version of batik.
The newer \texttt{net} then re-updates the same dependency to the 1.8pre version of batik.
It should be noted that in this compile, the 1.7 version of batik does not exist, and
is expected to be replaced by the 1.8pre version.

\begin{lstlisting}[caption=Replace]
//file net.n3.nanoxmlv2_2_1_4patch.module
module net.n3.nanoxmlv2_2_1_4patch 
	extends net.n3.nanoxml;

//file org.apache.batik1_6.module
module org.apache.batik1_6;
...

//file org.apache.batik1_8pre.module
module org.apache.batik1_8pre 
	overrides org.apache.batik1_6, org.apache.batik1_7;
...

//file org.jhotdraw.module
module org.jhotdraw;
import own org.apache.batik1_6 export as batik;
...

//file samples.defaultsample.module
module samples.defaultsample;
import own defaultsample.defaultjhotdraw 
	export as jhotdraw;
//replace nanoxml with a patched version
replace jhotdraw::nanoxml 
	with own net.n3.nanoxmlv2_2_1_4patch;
export package *;

//file samples.netold.module
module samples.netold extends samples.defaultsample;
//update to batik version 1.7, which is not in the build
replace jhotdraw::batik 
	with own org.apache.batik1_7;

//file samples.net.module
module samples.net extends samples.netold;
//update to batik version 1.8
replace jhotdraw::batik
	with own org.apache.batik1_8pre;
\end{lstlisting}

The example shows how a subtype module can replace a supertype module
reference, and how an overriding module can replace a overridden module
that is not part of the build.

In addition to the type rules for replace, it is also not allowed to 
replace a module reference that points to a singleton module or
is or at any point indirects through a singleton module. This is to ensure that other
modules that depend on the singleton instance can correctly keep their
assumptions about the unmodified state of the singleton.

%Merge and replace make different modules sure to use the same
%set of classes (avoids nondeterministic class loading). Also, can
%be used to update modules to use newer versions of their imported
%modules. They can also be used to access the namespace of indirect imports
%without qualifiers by merging the indirect imports with a directly
%imported instance of that module.


%------------------------------------------------------------------------- 
\SubSection{Merge}

It is often necessary to make sure that two or more modules use the same
instance of a dependency, to ensure that these
modules are able to share types. Furthermore, it should be possible for
the importers of a module to easily maintain this condition on a module's
instance. For this, we define the \textbf{merge} operation.

\begin{lstlisting}
merge m1,m2::m3,m4 [export] as type alias;
\end{lstlisting}

Similar to an \textbf{own} import declaration, a merge declaration creates
a module reference named \textbf{merge.alias} with type \textbf{type},
points this to a \textbf{own} instance of type type, and then replaces all
the module references in the list with the newly created instance. Furthermore,
any further replaces that target \texttt{alias} will also update all
the references in the list. This also works recursively: if \texttt{alias}
is targeted by another merge 
\begin{lstlisting}
merge alias, anotheralias export as newmerge;
\end{lstlisting}
then any replace that targets \texttt{newmerge} also update the references
in the merge declaration of \texttt{alias}.

The merge operation requires that the type of the created reference be replace
compatible with all the references in the merge list, as the instance will
have to be assigned to each of the references.

The following example taken from the JHotdraw case study shows \textbf{merge}
can be used to make sure that common dependencies between an importer and
an imported module can be made to point to the same module instance. The
application \texttt{samples.svg} imports the jhotdraw framework through
it's supertype module \texttt{samples.defaultsample}, and then merges
its own \texttt{batik} dependency with that of the imported \texttt{jhotdraw}.
Another application \texttt{samples.odg} imports \texttt{samples.svg}, 
and then merges its own instance of jhotdraw with that of the SVG application, 
while also updating the import's \texttt{batik} dependency with a newer version.

\begin{lstlisting}[caption=Merge]
//file org.jhotdraw.module
module org.jhotdraw;
import own org.apache.batik export as batik;
...

//file samples.svg.module
module samples.svg extends samples.defaultsample;
import own org.apache.batik1_6 export as batik;
...
//merge this module's batik import with
//that of the imported jhotdraw's
merge jhotdraw::batik, batik 
	export as org.apache.batik1_6 batik;
...

//file samples.odg.module
module samples.odg extends samples.defaultsample;
import own samples.svg as svg;
...
//merge this module's jhotdraw with svg's
//jhotdraw
merge svg::jhotdraw, jhotdraw 
	export as defaultsample.defaultjhotdraw jhotdraw;
//replace svg's batik with a new version
//this also updates jhotdraw's batik through
//the merge declaration in svg
replace svg::batik with own org.apache.batik1_8pre;
\end{lstlisting}


The merge in line 12 allows the clients of \texttt{samples.svg}s
the ability to update all of its \texttt{batik} dependencies
in a single replace declaration, as the \texttt{samples.odg}
application does in line 27. Similarly, the \texttt{samples.odg} application
uses merge in a similar manner in line 22, allowing any importers
to merge its own \texttt{jhotdraw} import with that of \texttt{samples.odg}.
The recursive nature of merge allows replaces to propagate through deep
import graphs of real applications.

Similar to replaces, merges are also implicitly inherited by subtype modules.

Once merged, a module reference can no longer be used in another merge. This
enforces the constraint that all the modules in the merge list point to the
same instance. Replace is still allowed to target the individual 
references in the merge, possibly pointing it to an instance different from
the other references. This is not advised, however, and will cause a warning to be
issued.

\SubSection{Module Interfaces}

Extending the object-oriented metaphor, with module types being classes, we 
now define \textit{module interfaces}, which act similarly to interfaces in Java. A module
interface contains no imports, replaces, merges or even member classes, but they
are allowed to have a set of export package declarations. A module implementing
a module interface must contain and export the packages specified in the interface.

\begin{lstlisting}[caption=Module Interfaces]
//file org.apache.batik.module
module_interface org.apache.batik;
export package 
	org.apache.batik.ext.awt.image,
	org.apache.batik.ext.awt,
	org.apache.batik;

//file org.jhotdraw.batikfragment.module
//this implements the interface above
module org.jhotdraw.batikfragment 
	implements org.apache.batik;
export package 
	org.apache.batik.ext.awt.image,
	org.apache.batik.ext.awt,
	org.apache.batik;

//file org.jhotdraw.module
module org.jhotdraw;
//import the interface instead of the module itself
import own org.apache.batik export as batik;
...
//replace the interface with an instance of
//the implementing module
replace batik with own org.jhotdraw.batikfragment; 
\end{lstlisting}

Rules for module interface subtyping are similar to those of interfaces in Java:
interfaces can only extend interfaces, and they can not implement other
interfaces. 

Since interfaces do not actually contain any classes, they must be replaced
by a non-interface module when compiling a fully working system. This would
have to be relaxed for separate compilation of modules that reference interfaces
to be possible, and this is expounded a bit more in section \ref{eval}.

\SubSection{Weak Module Interfaces}

It may be the case that an application developer knows the packages that
he wishes to use, but the provider of the module that contains these packages 
did not define a module interface that specifically contains those packages. For
this case, we define a \textit{weak module interface}. Weak module interfaces act similarly
to normal module interfaces, except that it is also implicitly implemented
by all modules that satisfy its export package signature \cite{mcdirmid01jiazzi, componentnextgen}, 
even if these modules did not explicitly declare that they implemented the interface.

The interface in the example above can be weakened, which allows the use
of the \texttt{batik} libraries without them explicitly implementing the interface.

\begin{lstlisting}[caption=Weak Interfaces]
//file org.apache.batik.module
weak_module_interface org.apache.batik;
export package org.apache.batik.ext.awt.image,
	org.apache.batik.ext.awt,	org.apache.batik;

//file org.jhotdraw.batikfragment.module
//no longer explicitly implements the interface
module org.jhotdraw.batikfragment;
export package org.apache.batik.ext.awt.image,
	org.apache.batik.ext.awt, org.apache.batik;
	
//file org.jhotdraw.module
module org.jhotdraw;
//import the weak interface 
import own org.apache.batik export as batik;
...
//replace the interface with an instance module
replace batik with own org.jhotdraw.batikfragment; 
\end{lstlisting}

%Type lookup subsection, excised for lack of space
%\subsection{Type Lookup Sequence}

This module system, as with any module system for Java, changes the way that type
references are looked up. The following pseudocode shows how type lookup is done in
the type system defined above. 

\begin{lstlisting}[caption = Type Lookup, tabsize=2, morekeywords={method}]
//lookup for unqualified names in a CU
method CU.lookup(typeName) {
	//get the module of which the CU is a member
	Module = CU.getParentModule();
	if (Module != null) {
		lookup classes in CU
		lookup classes in single type imports
		//lookup in module and its supermodules only
		Module.lookup(null, CU.package(), typeName, false)
		lookup classes in on-demand imports
		//lookup in module, including direct imports
		Module.lookup(null, CU.package(), typeName, true)
		lookup primitive types
		lookup automatic imports (java.lang)
	} else {
		normal java lookup
	}
}

//Lookup for qualified names. Takes the module qualifier, package
//qualifier and typeName of a type reference and a boolean value lookInImports
method Module.lookup(moduleName, packageName, 
							typeName, lookInImports) {
	if (special packageName (java.lang)) {
		lookup in defaultmodule
	}
	if (moduleName == null) {
		lookup in thismodule
		lookup in each successive supermodule
		if (lookInImports) {
			lookup in direct imports
		}
	} 
	else {
	//lookup through the module qualifier
	contextModule = lookupModule(moduleName);
	contextModule.lookup(``'', packageName, typeName, false);
	}
}
\end{lstlisting}

The lookup rules follow the Java way of looking up types,
starting from the most local proceeding to the most global. Types are looked up
first in the same compilation unit, then in the single type imports,
then in the types that belong to the same package in the module and its supermodules, 
then the on-demand imports, then the member types of the same package
in directly imported modules, and finally to the primitive types and the 
implicit {\tt java.lang} imports. It was a conscious decision to make on-demand imports
come first before the lookups to directly imported modules. This is because the on-demand import
is closer to the type reference being resolved, being in the same source file instead of
on a separate module specification file.


%section on constraints, now part of the indivudual sections where they apply
%\SubSection{Constraints on Modules}

%Cycles involving \texttt{import own} are not allowed, as this leads to an infinite loop of
%module instance creation. However, cycles of singleton imports are allowed.

%Subtyping is single inheritance, as in Java. This avoids ambiguous replace and
%merge inheritance order in a subtype module.

%Merges cannot be done on singleton instances or modules accessed through a
%singleton instance. This is to make sure that other clients using the singleton
%instance are not affected by the change.

%As with merges, replace targets can not be a singleton instance or a module
%accessed through a singleton instance, for similar reasons. 

%Merges can only be done on subtype modules in the same path to the root module in
%the subtype tree. Furthermore, this should not change the module export signature
%of the module being subtyped. This is to guarantee that both internal and external
%clients of the module can still rely on their module qualified lookups.

%Exported packages in non-interface modules imply that there are classes that actually 
%belong to that package. Otherwise module interfaces become less useful.

%As mentioned, implementing a module interface means exporting the packages the interface
%exports. This is to ensure that the signature contract with the clients of
%the interface are satisfied.

%All references to an interface module must be replaced. Otherwise all lookups through 
%that reference will fail as the interface is empty. 